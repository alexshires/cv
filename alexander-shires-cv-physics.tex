\documentclass[12pt,a4paper,sans]{moderncv} % possible options include font size
%('10pt', '11pt' and '12pt'), paper size ('a4paper', 'letterpaper', 'a5paper',
%'legalpaper', 'executivepaper' and 'landscape') and font family ('sans' and 'roman')

% moderncv themes
\moderncvstyle{banking} % style options are 'casual' (default), 'classic', 'oldstyle'
%and 'banking'
\moderncvcolor{blue} % color options 'blue' (default), 'orange', 'green', 'red',
%'purple', 'grey' and 'black'
%\renewcommand{\familydefault}{\sfdefault} % to set the default font; use
%'\sfdefault' for the default sans serif font, '\rmdefault' for the default roman
%one, or any tex font name
\nopagenumbers{} % uncomment to suppress automatic page numbering for CVs longer
%than one page

% character encoding
\usepackage[utf8]{inputenc} % if you are not using xelatex ou lualatex, replace by
%the encoding you are using
% adjust the page margins
%\usepackage[scale=0.75]{geometry}
\usepackage[scale=0.85]{geometry}
\setlength{\hintscolumnwidth}{2cm}
\AtBeginDocument{\recomputelengths}
%\setlength{\hintscolumnwidth}{3cm} % if you want to change the width of the column
%with the dates
%\setlength{\makecvtitlenamewidth}{10cm} % for the 'classic' style, if you want to
%force the width allocated to your name and avoid line breaks. be careful though, the
%length is normally calculated to avoid any overlap with your personal info; use this
%at your own typographical risks...

% personal data
\name{}{Dr Alexander Shires}
%\title{Alexander Shires' Resume} % optional, remove / comment the line if not
%wanted
\address{Ostenhellweg 56}{44135 Dortmund}{Germany}% optional, remove / comment the
%line if not wanted; the "postcode city" and "country" arguments can be omitted or
%provided empty
\phone[mobile]{+49~(173)~690~9175} % optional, remove / comment the line if not
%wanted; the optional "type" of the phone can be "mobile" (default), "fixed" or "fax"
\phone[fixed]{+44~(7799)~823~210}
%\phone[fax]{+3~(456)~789~012}
\email{a.shires@gmail.com} % optional, remove / comment the line if not wanted
%\homepage{www.johndoe.com} % optional, remove / comment the line if not wanted
%\social[linkedin]{alex-shires} % optional, remove / comment the line if not wanted
\social[twitter]{AlexanderShires} % optional, remove / comment the line if not
\social[github]{alexshires} % optional, remove / comment the line if not wanted
%\extrainfo{additional information} % optional, remove / comment the line if not
%\photo[64pt][0.4pt]{picture} % optional, remove / comment the line if not wanted;
%'64pt' is the height the picture must be resized to, 0.4pt is the thickness of the
%frame around it (put it to 0pt for no frame) and 'picture' is the name of the
%picture file
%\quote{Some quote} % optional, remove / comment the line if not wanted
%Publications
%\newcommand{\publication}[3]{\cventry{#1}{#2}{}{}{}{#3}}
\newcommand{\publication}[1]{\cvitem{}{#1}}

\def\tud{Technische Universit\"at Dortmund}
\def\icl{Imperial College London}
\def\name{Dr Alexander Shires}
% %%%%%%%%%%%%%%%%%%%%
%  for LHCb aliases
% %%%%%%%%%%%%%%%%%%%%
\usepackage{ifthen} % for conditional statements
\newboolean{pdflatex}
\setboolean{pdflatex}{true} % use this if using non-eps figures
\newboolean{articletitles}
\setboolean{articletitles}{true} % False removes titles in references
\newboolean{uprightparticles}
\setboolean{uprightparticles}{false} %Set to true to get roman particle symbols
\usepackage{amsmath} % Adds a large collection of math symbols
\usepackage{xspace} % To avoid problems with missing or double spaces after
% predefined symbold
\usepackage{bm}
\usepackage{amssymb}
\usepackage{amsfonts}
\usepackage{upgreek} % Adds in support for greek letters in roman typeset
%%% $Id: lhcb-symbols-def.tex 18993 2012-05-02 13:54:31Z uegede $
%%% ======================================================================
%%% Purpose: standard LHCb aliases
%%% Author: Originally Ulrik Egede, adapted by Tomasz Skwarnicki for templates,
%%% rewritten by Chris Parkes
%%% Created on: 2009-09-24
%%% =======================================================================

%%% this has to go before \begin{document}
%%%\usepackage{ifthen} 
%%%\newboolean{uprightparticles}
%%%\setboolean{uprightparticles}{true} %Set to false to get italic particle symbols

%%% Add comments with at least three %%% preceding.
%%% Add new sections with one % preceding
%%% Add new subsections with two %% preceding

%%%%%%%%%%%%%
% Experiments
%%%%%%%%%%%%%
\def\lhcb {\mbox{LHCb}\xspace}
\def\ux85 {\mbox{UX85}\xspace}
\def\cern {\mbox{CERN}\xspace}
\def\lhc    {\mbox{LHC}\xspace}
\def\atlas  {\mbox{ATLAS}\xspace}
\def\alice  {\mbox{ALICE}\xspace}
\def\cms    {\mbox{CMS}\xspace}
\def\babar  {\mbox{BaBar}\xspace}
\def\belle  {\mbox{Belle}\xspace}
\def\aleph  {\mbox{ALEPH}\xspace}
\def\delphi {\mbox{DELPHI}\xspace}
\def\opal   {\mbox{OPAL}\xspace}
\def\lthree {\mbox{L3}\xspace}
\def\lep    {\mbox{LEP}\xspace}
\def\cdf    {\mbox{CDF}\xspace}
\def\dzero  {\mbox{D0}\xspace}
\def\sld    {\mbox{SLD}\xspace}
\def\cleo   {\mbox{CLEO}\xspace}
\def\argus  {\mbox{ARGUS}\xspace}
\def\uaone  {\mbox{UA1}\xspace}
\def\uatwo  {\mbox{UA2}\xspace}
\def\tevatron {Tevatron\xspace}

%% LHCb sub-detectors and sub-systems

\def\pu     {PU\xspace}
\def\velo   {VELO\xspace}
\def\rich   {RICH\xspace}
\def\richone {RICH1\xspace}
\def\richtwo {RICH2\xspace}
\def\ttracker {TT\xspace}
\def\intr   {IT\xspace}
\def\st     {ST\xspace}
\def\ot     {OT\xspace}
\def\Tone   {T1\xspace}
\def\Ttwo   {T2\xspace}
\def\Tthree {T3\xspace}
\def\Mone   {M1\xspace}
\def\Mtwo   {M2\xspace}
\def\Mthree {M3\xspace}
\def\Mfour  {M4\xspace}
\def\Mfive  {M5\xspace}
\def\ecal   {ECAL\xspace}
\def\spd    {SPD\xspace}
\def\presh  {PS\xspace}
\def\hcal   {HCAL\xspace}
\def\bcm    {BCM\xspace}

\def\ode    {ODE\xspace}
\def\daq    {DAQ\xspace}
\def\tfc    {TFC\xspace}
\def\ecs    {ECS\xspace}
\def\lzero   {L0\xspace}
\def\hlt    {HLT\xspace}
\def\hltone {HLT1\xspace}
\def\hlttwo {HLT2\xspace}

%%% Upright (not slanted) Particles

\ifthenelse{\boolean{uprightparticles}}%
{\def\Palpha      {\ensuremath{\upalpha}\xspace}
 \def\Pbeta       {\ensuremath{\upbeta}\xspace}
 \def\Pgamma      {\ensuremath{\upgamma}\xspace}                 
 \def\Pdelta      {\ensuremath{\updelta}\xspace}                 
 \def\Pepsilon    {\ensuremath{\upepsilon}\xspace}                 
 \def\Pvarepsilon {\ensuremath{\upvarepsilon}\xspace}                 
 \def\Pzeta       {\ensuremath{\upzeta}\xspace}                 
 \def\Peta        {\ensuremath{\upeta}\xspace}                 
 \def\Ptheta      {\ensuremath{\uptheta}\xspace}                 
 \def\Pvartheta   {\ensuremath{\upvartheta}\xspace}                 
 \def\Piota       {\ensuremath{\upiota}\xspace}                 
 \def\Pkappa      {\ensuremath{\upkappa}\xspace}                 
 \def\Plambda     {\ensuremath{\uplambda}\xspace}                 
 \def\Pmu         {\ensuremath{\upmu}\xspace}                 
 \def\Pnu         {\ensuremath{\upnu}\xspace}                 
 \def\Pxi         {\ensuremath{\upxi}\xspace}                 
 \def\Ppi         {\ensuremath{\uppi}\xspace}                 
 \def\Pvarpi      {\ensuremath{\upvarpi}\xspace}                 
 \def\Prho        {\ensuremath{\uprho}\xspace}                 
 \def\Pvarrho     {\ensuremath{\upvarrho}\xspace}                 
 \def\Ptau        {\ensuremath{\uptau}\xspace}                 
 \def\Pupsilon    {\ensuremath{\upupsilon}\xspace}                 
 \def\Pphi        {\ensuremath{\upphi}\xspace}                 
 \def\Pvarphi     {\ensuremath{\upvarphi}\xspace}                 
 \def\Pchi        {\ensuremath{\upchi}\xspace}                 
 \def\Ppsi        {\ensuremath{\uppsi}\xspace}                 
 \def\Pomega      {\ensuremath{\upomega}\xspace}                 

 \def\PDelta      {\ensuremath{\Delta}\xspace}                 
 \def\PXi      {\ensuremath{\Xi}\xspace}                 
 \def\PLambda      {\ensuremath{\Lambda}\xspace}                 
 \def\PSigma      {\ensuremath{\Sigma}\xspace}                 
 \def\POmega      {\ensuremath{\Omega}\xspace}                 
 \def\PUpsilon      {\ensuremath{\Upsilon}\xspace}                 
 
 %\mathchardef\Deltares="7101
 %\mathchardef\Xi="7104
 %\mathchardef\Lambda="7103
 %\mathchardef\Sigma="7106
 %\mathchardef\Omega="710A


 \def\PA      {\ensuremath{\mathrm{A}}\xspace}                 
 \def\PB      {\ensuremath{\mathrm{B}}\xspace}                 
 \def\PC      {\ensuremath{\mathrm{C}}\xspace}                 
 \def\PD      {\ensuremath{\mathrm{D}}\xspace}                 
 \def\PE      {\ensuremath{\mathrm{E}}\xspace}                 
 \def\PF      {\ensuremath{\mathrm{F}}\xspace}                 
 \def\PG      {\ensuremath{\mathrm{G}}\xspace}                 
 \def\PH      {\ensuremath{\mathrm{H}}\xspace}                 
 \def\PI      {\ensuremath{\mathrm{I}}\xspace}                 
 \def\PJ      {\ensuremath{\mathrm{J}}\xspace}                 
 \def\PK      {\ensuremath{\mathrm{K}}\xspace}                 
 \def\PL      {\ensuremath{\mathrm{L}}\xspace}                 
 \def\PM      {\ensuremath{\mathrm{M}}\xspace}                 
 \def\PN      {\ensuremath{\mathrm{N}}\xspace}                 
 \def\PO      {\ensuremath{\mathrm{O}}\xspace}                 
 \def\PP      {\ensuremath{\mathrm{P}}\xspace}                 
 \def\PQ      {\ensuremath{\mathrm{Q}}\xspace}                 
 \def\PR      {\ensuremath{\mathrm{R}}\xspace}                 
 \def\PS      {\ensuremath{\mathrm{S}}\xspace}                 
 \def\PT      {\ensuremath{\mathrm{T}}\xspace}                 
 \def\PU      {\ensuremath{\mathrm{U}}\xspace}                 
 \def\PV      {\ensuremath{\mathrm{V}}\xspace}                 
 \def\PW      {\ensuremath{\mathrm{W}}\xspace}                 
 \def\PX      {\ensuremath{\mathrm{X}}\xspace}                 
 \def\PY      {\ensuremath{\mathrm{Y}}\xspace}                 
 \def\PZ      {\ensuremath{\mathrm{Z}}\xspace}                 
 \def\Pa      {\ensuremath{\mathrm{a}}\xspace}                 
 \def\Pb      {\ensuremath{\mathrm{b}}\xspace}                 
 \def\Pc      {\ensuremath{\mathrm{c}}\xspace}                 
 \def\Pd      {\ensuremath{\mathrm{d}}\xspace}                 
 \def\Pe      {\ensuremath{\mathrm{e}}\xspace}                 
 \def\Pf      {\ensuremath{\mathrm{f}}\xspace}                 
 \def\Pg      {\ensuremath{\mathrm{g}}\xspace}                 
 \def\Ph      {\ensuremath{\mathrm{h}}\xspace}                 
 \def\Pi      {\ensuremath{\mathrm{i}}\xspace}                 
 \def\Pj      {\ensuremath{\mathrm{j}}\xspace}                 
 \def\Pk      {\ensuremath{\mathrm{k}}\xspace}                 
 \def\Pl      {\ensuremath{\mathrm{l}}\xspace}                 
 \def\Pm      {\ensuremath{\mathrm{m}}\xspace}                 
 \def\Pn      {\ensuremath{\mathrm{n}}\xspace}                 
 \def\Po      {\ensuremath{\mathrm{o}}\xspace}                 
 \def\Pp      {\ensuremath{\mathrm{p}}\xspace}                 
 \def\Pq      {\ensuremath{\mathrm{q}}\xspace}                 
 \def\Pr      {\ensuremath{\mathrm{r}}\xspace}                 
 \def\Ps      {\ensuremath{\mathrm{s}}\xspace}                 
 \def\Pt      {\ensuremath{\mathrm{t}}\xspace}                 
 \def\Pu      {\ensuremath{\mathrm{u}}\xspace}                 
 \def\Pv      {\ensuremath{\mathrm{v}}\xspace}                 
 \def\Pw      {\ensuremath{\mathrm{w}}\xspace}                 
 \def\Px      {\ensuremath{\mathrm{x}}\xspace}                 
 \def\Py      {\ensuremath{\mathrm{y}}\xspace}                 
 \def\Pz      {\ensuremath{\mathrm{z}}\xspace}                 
}
{\def\Palpha      {\ensuremath{\alpha}\xspace}
 \def\Pbeta       {\ensuremath{\beta}\xspace}
 \def\Pgamma      {\ensuremath{\gamma}\xspace}                 
 \def\Pdelta      {\ensuremath{\delta}\xspace}                 
 \def\Pepsilon    {\ensuremath{\epsilon}\xspace}                 
 \def\Pvarepsilon {\ensuremath{\varepsilon}\xspace}                 
 \def\Pzeta       {\ensuremath{\zeta}\xspace}                 
 \def\Peta        {\ensuremath{\eta}\xspace}                 
 \def\Ptheta      {\ensuremath{\theta}\xspace}                 
 \def\Pvartheta   {\ensuremath{\vartheta}\xspace}                 
 \def\Piota       {\ensuremath{\iota}\xspace}                 
 \def\Pkappa      {\ensuremath{\kappa}\xspace}                 
 \def\Plambda     {\ensuremath{\lambda}\xspace}                 
 \def\Pmu         {\ensuremath{\mu}\xspace}                 
 \def\Pnu         {\ensuremath{\nu}\xspace}                 
 \def\Pxi         {\ensuremath{\xi}\xspace}                 
 \def\Ppi         {\ensuremath{\pi}\xspace}                 
 \def\Pvarpi      {\ensuremath{\varpi}\xspace}                 
 \def\Prho        {\ensuremath{\rho}\xspace}                 
 \def\Pvarrho     {\ensuremath{\varrho}\xspace}                 
 \def\Ptau        {\ensuremath{\tau}\xspace}                 
 \def\Pupsilon    {\ensuremath{\upsilon}\xspace}                 
 \def\Pphi        {\ensuremath{\phi}\xspace}                 
 \def\Pvarphi     {\ensuremath{\varphi}\xspace}                 
 \def\Pchi        {\ensuremath{\chi}\xspace}                 
 \def\Ppsi        {\ensuremath{\psi}\xspace}                 
 \def\Pomega      {\ensuremath{\omega}\xspace}                 
 \mathchardef\PDelta="7101
 \mathchardef\PXi="7104
 \mathchardef\PLambda="7103
 \mathchardef\PSigma="7106
 \mathchardef\POmega="710A
 \mathchardef\PUpsilon="7107
 \def\PA      {\ensuremath{A}\xspace}                 
 \def\PB      {\ensuremath{B}\xspace}                 
 \def\PC      {\ensuremath{C}\xspace}                 
 \def\PD      {\ensuremath{D}\xspace}                 
 \def\PE      {\ensuremath{E}\xspace}                 
 \def\PF      {\ensuremath{F}\xspace}                 
 \def\PG      {\ensuremath{G}\xspace}                 
 \def\PH      {\ensuremath{H}\xspace}                 
 \def\PI      {\ensuremath{I}\xspace}                 
 \def\PJ      {\ensuremath{J}\xspace}                 
 \def\PK      {\ensuremath{K}\xspace}                 
 \def\PL      {\ensuremath{L}\xspace}                 
 \def\PM      {\ensuremath{M}\xspace}                 
 \def\PN      {\ensuremath{N}\xspace}                 
 \def\PO      {\ensuremath{O}\xspace}                 
 \def\PP      {\ensuremath{P}\xspace}                 
 \def\PQ      {\ensuremath{Q}\xspace}                 
 \def\PR      {\ensuremath{R}\xspace}                 
 \def\PS      {\ensuremath{S}\xspace}                 
 \def\PT      {\ensuremath{T}\xspace}                 
 \def\PU      {\ensuremath{U}\xspace}                 
 \def\PV      {\ensuremath{V}\xspace}                 
 \def\PW      {\ensuremath{W}\xspace}                 
 \def\PX      {\ensuremath{X}\xspace}                 
 \def\PY      {\ensuremath{Y}\xspace}                 
 \def\PZ      {\ensuremath{Z}\xspace}                 
 \def\Pa      {\ensuremath{a}\xspace}                 
 \def\Pb      {\ensuremath{b}\xspace}                 
 \def\Pc      {\ensuremath{c}\xspace}                 
 \def\Pd      {\ensuremath{d}\xspace}                 
 \def\Pe      {\ensuremath{e}\xspace}                 
 \def\Pf      {\ensuremath{f}\xspace}                 
 \def\Pg      {\ensuremath{g}\xspace}                 
 \def\Ph      {\ensuremath{h}\xspace}                 
 \def\Pi      {\ensuremath{i}\xspace}                 
 \def\Pj      {\ensuremath{j}\xspace}                 
 \def\Pk      {\ensuremath{k}\xspace}                 
 \def\Pl      {\ensuremath{l}\xspace}                 
 \def\Pm      {\ensuremath{m}\xspace}                 
 \def\Pn      {\ensuremath{n}\xspace}                 
 \def\Po      {\ensuremath{o}\xspace}                 
 \def\Pp      {\ensuremath{p}\xspace}                 
 \def\Pq      {\ensuremath{q}\xspace}                 
 \def\Pr      {\ensuremath{r}\xspace}                 
 \def\Ps      {\ensuremath{s}\xspace}                 
 \def\Pt      {\ensuremath{t}\xspace}                 
 \def\Pu      {\ensuremath{u}\xspace}                 
 \def\Pv      {\ensuremath{v}\xspace}                 
 \def\Pw      {\ensuremath{w}\xspace}                 
 \def\Px      {\ensuremath{x}\xspace}                 
 \def\Py      {\ensuremath{y}\xspace}                 
 \def\Pz      {\ensuremath{z}\xspace}                 
}

%%%%%%%%%%%%%%%%%%%%%%%%%%%%%%%%%%%%%%%%%%%%%%%
% Particles

%% Leptons

\let\emi\en
\def\electron   {\ensuremath{\Pe}\xspace}
\def\en         {\ensuremath{\Pe^-}\xspace}   % electron negative (\em is taken)
\def\ep         {\ensuremath{\Pe^+}\xspace}
\def\epm        {\ensuremath{\Pe^\pm}\xspace} 
\def\epem       {\ensuremath{\Pe^+\Pe^-}\xspace}
\def\ee         {\ensuremath{\Pe^-\Pe^-}\xspace}

\def\mmu        {\ensuremath{\Pmu}\xspace}
\def\mup        {\ensuremath{\Pmu^+}\xspace}
\def\mun        {\ensuremath{\Pmu^-}\xspace} % muon negative (\mum is taken)
\def\mumu       {\ensuremath{\Pmu^+\Pmu^-}\xspace}
\def\mtau       {\ensuremath{\Ptau}\xspace}

\def\taup       {\ensuremath{\Ptau^+}\xspace}
\def\taum       {\ensuremath{\Ptau^-}\xspace}
\def\tautau     {\ensuremath{\Ptau^+\Ptau^-}\xspace}

\def\ellm       {\ensuremath{\ell^-}\xspace}
\def\ellp       {\ensuremath{\ell^+}\xspace}
\def\ellell     {\ensuremath{\ell^+ \ell^-}\xspace}

\def\neu        {\ensuremath{\Pnu}\xspace}
\def\neub       {\ensuremath{\overline{\Pnu}}\xspace}
\def\nuenueb    {\ensuremath{\neu\neub}\xspace}
\def\neue       {\ensuremath{\neu_e}\xspace}
\def\neueb      {\ensuremath{\neub_e}\xspace}
\def\neueneueb  {\ensuremath{\neue\neueb}\xspace}
\def\neum       {\ensuremath{\neu_\mu}\xspace}
\def\neumb      {\ensuremath{\neub_\mu}\xspace}
\def\neumneumb  {\ensuremath{\neum\neumb}\xspace}
\def\neut       {\ensuremath{\neu_\tau}\xspace}
\def\neutb      {\ensuremath{\neub_\tau}\xspace}
\def\neutneutb  {\ensuremath{\neut\neutb}\xspace}
\def\neul       {\ensuremath{\neu_\ell}\xspace}
\def\neulb      {\ensuremath{\neub_\ell}\xspace}
\def\neulneulb  {\ensuremath{\neul\neulb}\xspace}

%% Gauge bosons and scalars

\def\g      {\ensuremath{\Pgamma}\xspace}
\def\H      {\ensuremath{\PH^0}\xspace}
\def\Hp     {\ensuremath{\PH^+}\xspace}
\def\Hm     {\ensuremath{\PH^-}\xspace}
\def\Hpm    {\ensuremath{\PH^\pm}\xspace}
\def\W      {\ensuremath{\PW}\xspace}
\def\Wp     {\ensuremath{\PW^+}\xspace}
\def\Wm     {\ensuremath{\PW^-}\xspace}
\def\Wpm    {\ensuremath{\PW^\pm}\xspace}
\def\Z      {\ensuremath{\PZ^0}\xspace}

%% Quarks

\def\quark     {\ensuremath{\Pq}\xspace}
\def\quarkbar  {\ensuremath{\overline \quark}\xspace}
\def\qqbar     {\ensuremath{\quark\quarkbar}\xspace}
\def\uquark    {\ensuremath{\Pu}\xspace}
\def\uquarkbar {\ensuremath{\overline \uquark}\xspace}
\def\uubar     {\ensuremath{\uquark\uquarkbar}\xspace}
\def\dquark    {\ensuremath{\Pd}\xspace}
\def\dquarkbar {\ensuremath{\overline \dquark}\xspace}
\def\ddbar     {\ensuremath{\dquark\dquarkbar}\xspace}
\def\squark    {\ensuremath{\Ps}\xspace}
\def\squarkbar {\ensuremath{\overline \squark}\xspace}
\def\ssbar     {\ensuremath{\squark\squarkbar}\xspace}
\def\cquark    {\ensuremath{\Pc}\xspace}
\def\cquarkbar {\ensuremath{\overline \cquark}\xspace}
\def\ccbar     {\ensuremath{\cquark\cquarkbar}\xspace}
\def\bquark    {\ensuremath{\Pb}\xspace}
\def\bquarkbar {\ensuremath{\overline \bquark}\xspace}
\def\bbbar     {\ensuremath{\bquark\bquarkbar}\xspace}
\def\tquark    {\ensuremath{\Pt}\xspace}
\def\tquarkbar {\ensuremath{\overline \tquark}\xspace}
\def\ttbar     {\ensuremath{\tquark\tquarkbar}\xspace}

%% Light mesons

\def\pion  {\ensuremath{\Ppi}\xspace}
\def\piz   {\ensuremath{\pion^0}\xspace}
\def\pizs  {\ensuremath{\pion^0\mbox\,\rm{s}}\xspace}
\def\ppz   {\ensuremath{\pion^0\pion^0}\xspace}
\def\pip   {\ensuremath{\pion^+}\xspace}
\def\pim   {\ensuremath{\pion^-}\xspace}
\def\pipi  {\ensuremath{\pion^+\pion^-}\xspace}
\def\pipm  {\ensuremath{\pion^\pm}\xspace}
\def\pimp  {\ensuremath{\pion^\mp}\xspace}

\def\kaon  {\ensuremath{\PK}\xspace}
%%% do NOT use ensuremath here
  \def\Kbar  {\kern 0.2em\overline{\kern -0.2em \PK}{}\xspace}
\def\Kb    {\ensuremath{\Kbar}\xspace}
\def\Kz    {\ensuremath{\kaon^0}\xspace}
\def\Kzb   {\ensuremath{\Kbar^0}\xspace}
\def\KzKzb {\ensuremath{\Kz \kern -0.16em \Kzb}\xspace}
\def\Kp    {\ensuremath{\kaon^+}\xspace}
\def\Km    {\ensuremath{\kaon^-}\xspace}
\def\Kpm   {\ensuremath{\kaon^\pm}\xspace}
\def\Kmp   {\ensuremath{\kaon^\mp}\xspace}
\def\KpKm  {\ensuremath{\Kp \kern -0.16em \Km}\xspace}
\def\KS    {\ensuremath{\kaon^0_{\rm\scriptscriptstyle S}}\xspace} 
\def\KL    {\ensuremath{\kaon^0_{\rm\scriptscriptstyle L}}\xspace} 
\def\Kstarz  {\ensuremath{\kaon^{*0}}\xspace}
\def\Kstarzb {\ensuremath{\Kbar^{*0}}\xspace}
\def\Kstar   {\ensuremath{\kaon^*}\xspace}
\def\Kstarb  {\ensuremath{\Kbar^*}\xspace}
\def\Kstarp  {\ensuremath{\kaon^{*+}}\xspace}
\def\Kstarm  {\ensuremath{\kaon^{*-}}\xspace}
\def\Kstarpm {\ensuremath{\kaon^{*\pm}}\xspace}
\def\Kstarmp {\ensuremath{\kaon^{*\mp}}\xspace}

\newcommand{\etapr}{\ensuremath{\Peta^{\prime}}\xspace}

%% Heavy mesons

%%% do NOT use ensuremath here
  \def\Dbar    {\kern 0.2em\overline{\kern -0.2em \PD}{}\xspace}
\def\D       {\ensuremath{\PD}\xspace}
\def\Db      {\ensuremath{\Dbar}\xspace}
\def\Dz      {\ensuremath{\D^0}\xspace}
\def\Dzb     {\ensuremath{\Dbar^0}\xspace}
\def\DzDzb   {\ensuremath{\Dz {\kern -0.16em \Dzb}}\xspace}
\def\Dp      {\ensuremath{\D^+}\xspace}
\def\Dm      {\ensuremath{\D^-}\xspace}
\def\Dpm     {\ensuremath{\D^\pm}\xspace}
\def\Dmp     {\ensuremath{\D^\mp}\xspace}
\def\DpDm    {\ensuremath{\Dp {\kern -0.16em \Dm}}\xspace}
\def\Dstar   {\ensuremath{\D^*}\xspace}
\def\Dstarb  {\ensuremath{\Dbar^*}\xspace}
\def\Dstarz  {\ensuremath{\D^{*0}}\xspace}
\def\Dstarzb {\ensuremath{\Dbar^{*0}}\xspace}
\def\Dstarp  {\ensuremath{\D^{*+}}\xspace}
\def\Dstarm  {\ensuremath{\D^{*-}}\xspace}
\def\Dstarpm {\ensuremath{\D^{*\pm}}\xspace}
\def\Dstarmp {\ensuremath{\D^{*\mp}}\xspace}
\def\Ds      {\ensuremath{\D^+_\squark}\xspace}
\def\Dsp     {\ensuremath{\D^+_\squark}\xspace}
\def\Dsm     {\ensuremath{\D^-_\squark}\xspace}
\def\Dspm    {\ensuremath{\D^{\pm}_\squark}\xspace}
\def\Dss     {\ensuremath{\D^{*+}_\squark}\xspace}
\def\Dssp    {\ensuremath{\D^{*+}_\squark}\xspace}
\def\Dssm    {\ensuremath{\D^{*-}_\squark}\xspace}
\def\Dsspm   {\ensuremath{\D^{*\pm}_\squark}\xspace}

\def\B       {\ensuremath{\PB}\xspace}
%%% do NOT use ensuremath here
  \def\Bbar    {\kern 0.18em\overline{\kern -0.18em \PB}{}\xspace}
\def\Bb      {\ensuremath{\Bbar}\xspace}
\def\BBbar   {\ensuremath{\B\Bbar}\xspace} 
\def\Bz      {\ensuremath{\B^0}\xspace}
\def\Bzb     {\ensuremath{\Bbar^0}\xspace}
\def\Bu      {\ensuremath{\B^+}\xspace}
\def\Bub     {\ensuremath{\B^-}\xspace}
\def\Bp      {\ensuremath{\Bu}\xspace}
\def\Bm      {\ensuremath{\Bub}\xspace}
\def\Bpm     {\ensuremath{\B^\pm}\xspace}
\def\Bmp     {\ensuremath{\B^\mp}\xspace}
\def\Bd      {\ensuremath{\B^0}\xspace}
\def\Bs      {\ensuremath{\B^0_\squark}\xspace}
\def\Bsb     {\ensuremath{\Bbar^0_\squark}\xspace}
\def\Bdb     {\ensuremath{\Bbar^0}\xspace}
\def\Bc      {\ensuremath{\B_\cquark^+}\xspace}
\def\Bcp     {\ensuremath{\B_\cquark^+}\xspace}
\def\Bcm     {\ensuremath{\B_\cquark^-}\xspace}
\def\Bcpm    {\ensuremath{\B_\cquark^\pm}\xspace}

%% Onia

\def\jpsi     {\ensuremath{{\PJ\mskip -3mu/\mskip -2mu\Ppsi\mskip 2mu}}\xspace}
\def\psitwos  {\ensuremath{\Ppsi{(2S)}}\xspace}
\def\psiprpr  {\ensuremath{\Ppsi(3770)}\xspace}
\def\etac     {\ensuremath{\Peta_\cquark}\xspace}
\def\chiczero {\ensuremath{\Pchi_{\cquark 0}}\xspace}
\def\chicone  {\ensuremath{\Pchi_{\cquark 1}}\xspace}
\def\chictwo  {\ensuremath{\Pchi_{\cquark 2}}\xspace}
  %\mathchardef\Upsilon="7107
  \def\Y#1S{\ensuremath{\PUpsilon{(#1S)}}\xspace}% no space before {...}!
\def\OneS  {\Y1S}
\def\TwoS  {\Y2S}
\def\ThreeS{\Y3S}
\def\FourS {\Y4S}
\def\FiveS {\Y5S}

\def\chic  {\ensuremath{\Pchi_{c}}\xspace}

%% Baryons

\def\proton      {\ensuremath{\Pp}\xspace}
\def\antiproton  {\ensuremath{\overline \proton}\xspace}
\def\neutron     {\ensuremath{\Pn}\xspace}
\def\antineutron {\ensuremath{\overline \neutron}\xspace}

\def\Deltares {\ensuremath{\PDelta}\xspace}
\def\Deltaresbar{\ensuremath{\overline \Deltares}\xspace}
\def\Xires {\ensuremath{\PXi}\xspace}
\def\Xiresbar{\ensuremath{\overline \Xires}\xspace}
\def\L {\ensuremath{\PLambda}\xspace}
\def\Lbar {\ensuremath{\kern 0.1em\overline{\kern -0.1em\PLambda}}\xspace}
\def\Lambdares {\ensuremath{\PLambda}\xspace}
\def\Lambdaresbar{\ensuremath{\Lbar}\xspace}
\def\Sigmares {\ensuremath{\PSigma}\xspace}
\def\Sigmaresbar{\ensuremath{\overline \Sigmares}\xspace}
\def\Omegares {\ensuremath{\POmega}\xspace}
\def\Omegaresbar{\ensuremath{\overline \Omegares}\xspace}

%%% do NOT use ensuremath here
 % \def\Deltabar{\kern 0.25em\overline{\kern -0.25em \Deltares}{}\xspace}
 % \def\Sigbar{\kern 0.2em\overline{\kern -0.2em \Sigma}{}\xspace}
 % \def\Xibar{\kern 0.2em\overline{\kern -0.2em \Xi}{}\xspace}
 % \def\Obar{\kern 0.2em\overline{\kern -0.2em \Omega}{}\xspace}
 % \def\Nbar{\kern 0.2em\overline{\kern -0.2em N}{}\xspace}
 % \def\Xb{\kern 0.2em\overline{\kern -0.2em X}{}\xspace}

\def\Lb      {\ensuremath{\L^0_\bquark}\xspace}
\def\Lbbar   {\ensuremath{\Lbar^0_\bquark}\xspace}
\def\Lc      {\ensuremath{\L^+_\cquark}\xspace}
\def\Lcbar   {\ensuremath{\Lbar^-_\cquark}\xspace}

%%%%%%%%%%%%%%%%%%
% Physics symbols
%%%%%%%%%%%%%%%%%

%% Decays
\def\BF         {{\ensuremath{\cal B}\xspace}}
\def\BRvis      {{\ensuremath{\BR_{\rm{vis}}}}}
\def\BR         {\BF}
\newcommand{\decay}[2]{\ensuremath{#1\!\to #2}\xspace}         % {\Pa}{\Pb \Pc}
\def\ra                 {\ensuremath{\rightarrow}\xspace}
\def\to                 {\ensuremath{\rightarrow}\xspace}

%% Lifetimes
\newcommand{\tauBs}{\ensuremath{\tau_{\Bs}}\xspace}
\newcommand{\tauBd}{\ensuremath{\tau_{\Bd}}\xspace}
\newcommand{\tauBz}{\ensuremath{\tau_{\Bz}}\xspace}
\newcommand{\tauBu}{\ensuremath{\tau_{\Bp}}\xspace}
\newcommand{\tauDp}{\ensuremath{\tau_{\Dp}}\xspace}
\newcommand{\tauDz}{\ensuremath{\tau_{\Dz}}\xspace}
\newcommand{\tauL}{\ensuremath{\tau_{\rm L}}\xspace}
\newcommand{\tauH}{\ensuremath{\tau_{\rm H}}\xspace}

%% Masses
\newcommand{\mBd}{\ensuremath{m_{\Bd}}\xspace}
\newcommand{\mBp}{\ensuremath{m_{\Bp}}\xspace}
\newcommand{\mBs}{\ensuremath{m_{\Bs}}\xspace}
\newcommand{\mBc}{\ensuremath{m_{\Bc}}\xspace}
\newcommand{\mLb}{\ensuremath{m_{\Lb}}\xspace}

%% EW theory, groups
\def\grpsuthree {\ensuremath{\mathrm{SU}(3)}\xspace}
\def\grpsutw    {\ensuremath{\mathrm{SU}(2)}\xspace}
\def\grpuone    {\ensuremath{\mathrm{U}(1)}\xspace}

\def\ssqtw {\ensuremath{\sin^{2}\!\theta_{\mathrm{W}}}\xspace}
\def\csqtw {\ensuremath{\cos^{2}\!\theta_{\mathrm{W}}}\xspace}
\def\stw   {\ensuremath{\sin\theta_{\mathrm{W}}}\xspace}
\def\ctw   {\ensuremath{\cos\theta_{\mathrm{W}}}\xspace}
\def\ssqtwef {\ensuremath{{\sin}^{2}\theta_{\mathrm{W}}^{\mathrm{eff}}}\xspace}
\def\csqtwef {\ensuremath{{\cos}^{2}\theta_{\mathrm{W}}^{\mathrm{eff}}}\xspace}
\def\stwef {\ensuremath{\sin\theta_{\mathrm{W}}^{\mathrm{eff}}}\xspace}
\def\ctwef {\ensuremath{\cos\theta_{\mathrm{W}}^{\mathrm{eff}}}\xspace}
\def\gv    {\ensuremath{g_{\mbox{\tiny V}}}\xspace}
\def\ga    {\ensuremath{g_{\mbox{\tiny A}}}\xspace}

\def\order   {\ensuremath{\mathcal{O}}\xspace}
\def\ordalph {\ensuremath{\mathcal{O}(\alpha)}\xspace}
\def\ordalsq {\ensuremath{\mathcal{O}(\alpha^{2})}\xspace}
\def\ordalcb {\ensuremath{\mathcal{O}(\alpha^{3})}\xspace}

%% QCD parameters
\newcommand{\as}{\ensuremath{\alpha_{\scriptscriptstyle S}}\xspace}
\newcommand{\MSb}{\ensuremath{\overline{\mathrm{MS}}}\xspace}
\newcommand{\lqcd}{\ensuremath{\Lambda_{\mathrm{QCD}}}\xspace}
\def\qsq       {\ensuremath{q^2}\xspace}

%% CKM, CP violation

\def\ckm   {\ensuremath{\mathrm{CKM}}\xspace}

\def\eps   {\ensuremath{\varepsilon}\xspace}
\def\epsK  {\ensuremath{\varepsilon_K}\xspace}
\def\epsB  {\ensuremath{\varepsilon_B}\xspace}
\def\epsp  {\ensuremath{\varepsilon^\prime_K}\xspace}

\def\CP                {\ensuremath{C\!P}\xspace}
\def\CPT               {\ensuremath{C\!PT}\xspace}

\def\rhobar {\ensuremath{\overline \rho}\xspace}
\def\etabar {\ensuremath{\overline \eta}\xspace}

\def\Vud  {\ensuremath{|V_{\uquark\dquark}|}\xspace}
\def\Vcd  {\ensuremath{|V_{\cquark\dquark}|}\xspace}
\def\Vtd  {\ensuremath{|V_{\tquark\dquark}|}\xspace}
\def\Vus  {\ensuremath{|V_{\uquark\squark}|}\xspace}
\def\Vcs  {\ensuremath{|V_{\cquark\squark}|}\xspace}
\def\Vts  {\ensuremath{|V_{\tquark\squark}|}\xspace}
\def\Vub  {\ensuremath{|V_{\uquark\bquark}|}\xspace}
\def\Vcb  {\ensuremath{|V_{\cquark\bquark}|}\xspace}
\def\Vtb  {\ensuremath{|V_{\tquark\bquark}|}\xspace}

%% Oscillations

\newcommand{\dm}{\ensuremath{\Delta m}\xspace}
\newcommand{\dms}{\ensuremath{\Delta m_{\squark}}\xspace}
\newcommand{\dmd}{\ensuremath{\Delta m_{\dquark}}\xspace}
\newcommand{\DG}{\ensuremath{\Delta\Gamma}\xspace}
\newcommand{\DGs}{\ensuremath{\Delta\Gamma_{\squark}}\xspace}
\newcommand{\DGd}{\ensuremath{\Delta\Gamma_{\dquark}}\xspace}
\newcommand{\Gs}{\ensuremath{\Gamma_{\squark}}\xspace}
\newcommand{\Gd}{\ensuremath{\Gamma_{\dquark}}\xspace}

\newcommand{\MBq}{\ensuremath{M_{\B_\quark}}\xspace}
\newcommand{\DGq}{\ensuremath{\Delta\Gamma_{\quark}}\xspace}
\newcommand{\Gq}{\ensuremath{\Gamma_{\quark}}\xspace}
\newcommand{\dmq}{\ensuremath{\Delta m_{\quark}}\xspace}
\newcommand{\GL}{\ensuremath{\Gamma_{\rm L}}\xspace}
\newcommand{\GH}{\ensuremath{\Gamma_{\rm H}}\xspace}

\newcommand{\DGsGs}{\ensuremath{\Delta\Gamma_{\squark}/\Gamma_{\squark}}\xspace}
\newcommand{\Delm}{\mbox{$\Delta m $}\xspace}
\newcommand{\ACP}{\ensuremath{{\cal A}^{\CP}}\xspace}
\newcommand{\Adir}{\ensuremath{{\cal A}^{\rm dir}}\xspace}
\newcommand{\Amix}{\ensuremath{{\cal A}^{\rm mix}}\xspace}
\newcommand{\ADelta}{\ensuremath{{\cal A}^\Delta}\xspace}
\newcommand{\phid}{\ensuremath{\phi_{\dquark}}\xspace}
\newcommand{\sinphid}{\ensuremath{\sin\!\phid}\xspace}
\newcommand{\phis}{\ensuremath{\phi_{\squark}}\xspace}
\newcommand{\betas}{\ensuremath{\beta_{\squark}}\xspace}
\newcommand{\sbetas}{\ensuremath{\sigma(\beta_{\squark})}\xspace}
\newcommand{\stbetas}{\ensuremath{\sigma(2\beta_{\squark})}\xspace}
\newcommand{\stphis}{\ensuremath{\sigma(\phi_{\squark})}\xspace}
\newcommand{\sinphis}{\ensuremath{\sin\!\phis}\xspace}

%% Tagging
\newcommand{\edet}{{\ensuremath{\varepsilon_{\rm det}}}\xspace}
\newcommand{\erec}{{\ensuremath{\varepsilon_{\rm rec/det}}}\xspace}
\newcommand{\esel}{{\ensuremath{\varepsilon_{\rm sel/rec}}}\xspace}
\newcommand{\etrg}{{\ensuremath{\varepsilon_{\rm trg/sel}}}\xspace}
\newcommand{\etot}{{\ensuremath{\varepsilon_{\rm tot}}}\xspace}

\newcommand{\mistag}{\ensuremath{\omega}\xspace}
\newcommand{\wcomb}{\ensuremath{\omega^{\rm comb}}\xspace}
\newcommand{\etag}{{\ensuremath{\varepsilon_{\rm tag}}}\xspace}
\newcommand{\etagcomb}{{\ensuremath{\varepsilon_{\rm tag}^{\rm comb}}}\xspace}
\newcommand{\effeff}{\ensuremath{\varepsilon_{\rm eff}}\xspace}
\newcommand{\effeffcomb}{\ensuremath{\varepsilon_{\rm eff}^{\rm comb}}\xspace}
\newcommand{\efftag}{{\ensuremath{\etag(1-2\omega)^2}}\xspace}
\newcommand{\effD}{{\ensuremath{\etag D^2}}\xspace}

\newcommand{\etagprompt}{{\ensuremath{\varepsilon_{\rm tag}^{\rm Pr}}}\xspace}
\newcommand{\etagLL}{{\ensuremath{\varepsilon_{\rm tag}^{\rm LL}}}\xspace}

%% Key decay channels

\def\BdToKstmm    {\decay{\Bd}{\Kstarz\mup\mun}}
\def\BdbToKstmm   {\decay{\Bdb}{\Kstarzb\mup\mun}}

\def\BuKll    {\decay{\Bu}{\Kp\ellell}}
\def\BdKstmm    {\decay{\Bd}{\Kstarz\mup\mun}}

\def\btosll     {\decay{\bquark}{\squark \ell^+ \ell^-}}
\def\btodll     {\decay{\bquark}{\dquark \ell^+ \ell^-}}

\def\BsToJPsiPhi  {\decay{\Bs}{\jpsi\phi}}
\def\BdToJPsiKst  {\decay{\Bd}{\jpsi\Kstarz}}
\def\BdbToJPsiKst {\decay{\Bdb}{\jpsi\Kstarzb}}

\def\BsPhiGam     {\decay{\Bs}{\phi \g}}
\def\BdKstGam     {\decay{\Bd}{\Kstarz \g}}

\def\BTohh        {\decay{\B}{\Ph^+ \Ph'^-}}
\def\BdTopipi     {\decay{\Bd}{\pip\pim}}
\def\BdToKpi      {\decay{\Bd}{\Kp\pim}}
\def\BsToKK       {\decay{\Bs}{\Kp\Km}}
\def\BsTopiK      {\decay{\Bs}{\pip\Km}}

%% Rare decays
\def\BdKstee  {\decay{\Bd}{\Kstarz\epem}}
\def\BdbKstee {\decay{\Bdb}{\Kstarzb\epem}}
\def\bsll     {\decay{\bquark}{\squark \ell^+ \ell^-}}
\def\AFB      {\ensuremath{A_{\mathrm{FB}}}\xspace}
\def\FL       {\ensuremath{F_{\mathrm{L}}}\xspace}
\def\AT#1     {\ensuremath{A_{\mathrm{T}}^{#1}}\xspace}           % 2
\def\ATIm     {\ensuremath{A_{\mathrm{T}}^{\text{Im}}}\xspace}           % 2
\def\ATRe     {\ensuremath{A_{\mathrm{T}}^{\text{Re}}}\xspace}           % 2
\def\btosgam  {\decay{\bquark}{\squark \g}}
\def\btodgam  {\decay{\bquark}{\dquark \g}}
\def\Bsmm     {\decay{\Bs}{\mup\mun}}
\def\Bdmm     {\decay{\Bd}{\mup\mun}}
\def\ctl       {\ensuremath{\cos{\theta_l}}\xspace}
\def\ctk       {\ensuremath{\cos{\theta_K}}\xspace}

%% Wilson coefficients and operators
\def\C#1      {\ensuremath{\mathcal{C}_{#1}}\xspace}                       % 9
\def\Cp#1     {\ensuremath{\mathcal{C}_{#1}^{'}}\xspace}                    % 7
\def\Ceff#1   {\ensuremath{\mathcal{C}_{#1}^{\mathrm{(eff)}}}\xspace}        % 9  
\def\Cpeff#1  {\ensuremath{\mathcal{C}_{#1}^{'\mathrm{(eff)}}}\xspace}       % 7
\def\Ope#1    {\ensuremath{\mathcal{O}_{#1}}\xspace}                       % 2
\def\Opep#1   {\ensuremath{\mathcal{O}_{#1}^{'}}\xspace}                    % 7

%% Charm

\def\xprime     {\ensuremath{x^{\prime}}\xspace}
\def\yprime     {\ensuremath{y^{\prime}}\xspace}
\def\ycp        {\ensuremath{y_{\CP}}\xspace}
\def\agamma     {\ensuremath{A_{\Gamma}}\xspace}
\def\kpi        {\ensuremath{\PK\Ppi}\xspace}
\def\kk         {\ensuremath{\PK\PK}\xspace}
\def\dkpi       {\decay{\PD}{\PK\Ppi}}
\def\dkk        {\decay{\PD}{\PK\PK}}
\def\dkpicf     {\decay{\Dz}{\Km\pip}}

%% QM
\newcommand{\bra}[1]{\ensuremath{\langle #1|}}             % {a}
\newcommand{\ket}[1]{\ensuremath{|#1\rangle}}              % {b}
\newcommand{\braket}[2]{\ensuremath{\langle #1|#2\rangle}} % {a}{b}

%%%%%%%%%%%%%%%%%%%%%%%%%%%%%%%%%%%%%%%%%%%%%%%%%%
% Units
%%%%%%%%%%%%%%%%%%%%%%%%%%%%%%%%%%%%%%%%%%%%%%%%%%
\newcommand{\unit}[1]{\ensuremath{\rm\,#1}\xspace}          % {kg}

%% Energy and momentum
\newcommand{\tev}{\ensuremath{\mathrm{\,Te\kern -0.1em V}}\xspace}
\newcommand{\gev}{\ensuremath{\mathrm{\,Ge\kern -0.1em V}}\xspace}
\newcommand{\mev}{\ensuremath{\mathrm{\,Me\kern -0.1em V}}\xspace}
\newcommand{\kev}{\ensuremath{\mathrm{\,ke\kern -0.1em V}}\xspace}
\newcommand{\ev}{\ensuremath{\mathrm{\,e\kern -0.1em V}}\xspace}
\newcommand{\gevc}{\ensuremath{{\mathrm{\,Ge\kern -0.1em V\!/}c}}\xspace}
\newcommand{\mevc}{\ensuremath{{\mathrm{\,Me\kern -0.1em V\!/}c}}\xspace}
\newcommand{\gevcc}{\ensuremath{{\mathrm{\,Ge\kern -0.1em V\!/}c^2}}\xspace}
\newcommand{\gevgevcccc}{\ensuremath{{\mathrm{\,Ge\kern -0.1em V^2\!/}c^4}}\xspace}
\newcommand{\mevcc}{\ensuremath{{\mathrm{\,Me\kern -0.1em V\!/}c^2}}\xspace}

%% Distance and area
\def\km   {\ensuremath{\rm \,km}\xspace}
\def\m    {\ensuremath{\rm \,m}\xspace}
\def\cm   {\ensuremath{\rm \,cm}\xspace}
\def\cma  {\ensuremath{{\rm \,cm}^2}\xspace}
\def\mm   {\ensuremath{\rm \,mm}\xspace}
\def\mma  {\ensuremath{{\rm \,mm}^2}\xspace}
\def\mum  {\ensuremath{\,\upmu\rm m}\xspace}
\def\muma {\ensuremath{\,\upmu\rm m^2}\xspace}
\def\nm   {\ensuremath{\rm \,nm}\xspace}
\def\fm   {\ensuremath{\rm \,fm}\xspace}
\def\barn{\ensuremath{\rm \,b}\xspace}
\def\barnhyph{\ensuremath{\rm -b}\xspace}
\def\mbarn{\ensuremath{\rm \,mb}\xspace}
\def\mub{\ensuremath{\rm \,\upmu b}\xspace}
\def\mbarnhyph{\ensuremath{\rm -mb}\xspace}
\def\nb {\ensuremath{\rm \,nb}\xspace}
\def\invnb {\ensuremath{\mbox{\,nb}^{-1}}\xspace}
\def\pb {\ensuremath{\rm \,pb}\xspace}
\def\invpb {\ensuremath{\mbox{\,pb}^{-1}}\xspace}
\def\fb   {\ensuremath{\mbox{\,fb}}\xspace}
\def\invfb   {\ensuremath{\mbox{\,fb}^{-1}}\xspace}

%% Time 
\def\sec  {\ensuremath{\rm {\,s}}\xspace}
\def\ms   {\ensuremath{{\rm \,ms}}\xspace}
\def\mus  {\ensuremath{\,\upmu{\rm s}}\xspace}
\def\ns   {\ensuremath{{\rm \,ns}}\xspace}
\def\ps   {\ensuremath{{\rm \,ps}}\xspace}
\def\fs   {\ensuremath{\rm \,fs}\xspace}

\def\mhz  {\ensuremath{{\rm \,MHz}}\xspace}
\def\khz  {\ensuremath{{\rm \,kHz}}\xspace}
\def\hz   {\ensuremath{{\rm \,Hz}}\xspace}

\def\invps{\ensuremath{{\rm \,ps^{-1}}}\xspace}

\def\yr   {\ensuremath{\rm \,yr}\xspace}
\def\hr   {\ensuremath{\rm \,hr}\xspace}

%% Temperature
\def\degc {\ensuremath{^\circ}{C}\xspace}
\def\degk {\ensuremath {\rm K}\xspace}

%% Material lengths, radiation
\def\Xrad {\ensuremath{X_0}\xspace}
\def\NIL{\ensuremath{\lambda_{int}}\xspace}
\def\mip {MIP\xspace}
\def\neutroneq {\ensuremath{\rm \,n_{eq}}\xspace}
\def\neqcmcm {\ensuremath{\rm \,n_{eq} / cm^2}\xspace}
\def\kRad {\ensuremath{\rm \,kRad}\xspace}
\def\MRad {\ensuremath{\rm \,MRad}\xspace}
\def\ci {\ensuremath{\rm \,Ci}\xspace}
\def\mci {\ensuremath{\rm \,mCi}\xspace}

%% Uncertainties
\def\sx    {\ensuremath{\sigma_x}\xspace}    
\def\sy    {\ensuremath{\sigma_y}\xspace}   
\def\sz    {\ensuremath{\sigma_z}\xspace}    

\newcommand{\stat}{\ensuremath{\mathrm{(stat)}}\xspace}
\newcommand{\syst}{\ensuremath{\mathrm{(syst)}}\xspace}

%% Maths

\def\order{{\ensuremath{\cal O}}\xspace}
\newcommand{\chisq}{\ensuremath{\chi^2}\xspace}

\def\deriv {\ensuremath{\mathrm{d}}}

\def\gsim{{~\raise.15em\hbox{$>$}\kern-.85em
          \lower.35em\hbox{$\sim$}~}\xspace}
\def\lsim{{~\raise.15em\hbox{$<$}\kern-.85em
          \lower.35em\hbox{$\sim$}~}\xspace}

\newcommand{\mean}[1]{\ensuremath{\left\langle #1 \right\rangle}} % {x}
\newcommand{\abs}[1]{\ensuremath{\left\|#1\right\|}} % {x}
\newcommand{\Real}{\ensuremath{\mathcal{R}e}\xspace}
\newcommand{\Imag}{\ensuremath{\mathcal{I}m}\xspace}

\def\PDF {PDF\xspace}
\def\PDFs {PDFs\xspace}

\def\sPlot{\mbox{\em sPlot}}
\def\sWeights{\mbox{\em sWeights}}
%%%%%%%%%%%%%%%%%%%%%%%%%%%%%%%%%%%%%%%%%%%%%%%%%%
% Kinematics
%%%%%%%%%%%%%%%%%%%%%%%%%%%%%%%%%%%%%%%%%%%%%%%%%%

%% Energy, Momenta
\def\Ebeam {\ensuremath{E_{\mbox{\tiny BEAM}}}\xspace}
\def\sqs   {\ensuremath{\protect\sqrt{s}}\xspace}

\def\ptot       {\mbox{$p$}\xspace}
\def\pt         {\mbox{$p_{\rm T}$}\xspace}
\def\et         {\mbox{$E_{\rm T}$}\xspace}
\def\dpp        {\ensuremath{\mathrm{d}\hspace{-0.1em}p/p}\xspace}

\newcommand{\dedx}{\ensuremath{\mathrm{d}\hspace{-0.1em}E/\mathrm{d}x}\xspace}

%% PID

\def\dll     {\ensuremath{\mathrm{\Delta(\log\mathcal{L})}}\xspace}
\def\dllkpi     {\ensuremath{\mathrm{\Delta(\log\mathcal{L})}_{\kaon\pion}}\xspace}
\def\dllppi     {\ensuremath{\mathrm{\Delta(\log\mathcal{L})}_{\proton\pion}}\xspace}
\def\dllepi     {\ensuremath{\mathrm{\Delta(\log\mathcal{L})}_{\electron\pion}}\xspace}
\def\dllmupi    {\ensuremath{\mathrm{\Delta(\log\mathcal{L})}_{\mmu\pi}}\xspace}

%% Geometry
\def\mphi       {\mbox{$\phi$}\xspace}
\def\mtheta     {\mbox{$\theta$}\xspace}
\def\ctheta     {\mbox{$\cos\theta$}\xspace}
\def\stheta     {\mbox{$\sin\theta$}\xspace}
\def\ttheta     {\mbox{$\tan\theta$}\xspace}

\def\degrees{\ensuremath{^{\circ}}\xspace}
\def\krad {\ensuremath{\rm \,krad}\xspace}
\def\mrad{\ensuremath{\rm \,mrad}\xspace}
\def\rad{\ensuremath{\rm \,rad}\xspace}

%% Accelerator
\def\betastar {\ensuremath{\beta^*}}
\newcommand{\lum} {\ensuremath{\mathcal{L}}\xspace}
\newcommand{\intlum}[1]{\ensuremath{\int\lum=#1\xspace}}  % {2 \,\invfb}

%%%%%%%%%%%%%%%%%%%%%%%%%%%%%%%%%%%%%%%%%%%%%%%%%%%%%%%%%%%%%%%%%%%%
% Software
%%%%%%%%%%%%%%%%%%%%%%%%%%%%%%%%%%%%%%%%%%%%%%%%%%%%%%%%%%%%%%%%%%%%

%% Programs
\def\evtgen     {\mbox{\textsc{EvtGen}}\xspace}
\def\pythia     {\mbox{\textsc{Pythia}}\xspace}
\def\fluka      {\mbox{\textsc{Fluka}}\xspace}
\def\tosca      {\mbox{\textsc{Tosca}}\xspace}
\def\ansys      {\mbox{\textsc{Ansys}}\xspace}
\def\spice      {\mbox{\textsc{Spice}}\xspace}
\def\garfield   {\mbox{\textsc{Garfield}}\xspace}
\def\geant      {\mbox{\textsc{Geant4}}\xspace}
\def\hepmc      {\mbox{\textsc{HepMC}}\xspace}
\def\gauss      {\mbox{\textsc{Gauss}}\xspace}
\def\gaudi      {\mbox{\textsc{Gaudi}}\xspace}
\def\boole      {\mbox{\textsc{Boole}}\xspace}
\def\brunel     {\mbox{\textsc{Brunel}}\xspace}
\def\davinci    {\mbox{\textsc{DaVinci}}\xspace}
\def\erasmus    {\mbox{\textsc{Erasmus}}\xspace}
\def\moore      {\mbox{\textsc{Moore}}\xspace}
\def\ganga      {\mbox{\textsc{Ganga}}\xspace}
\def\dirac      {\mbox{\textsc{Dirac}}\xspace}
\def\root       {\mbox{\textsc{Root}}\xspace}
\def\roofit     {\mbox{\textsc{RooFit}}\xspace}
\def\pyroot     {\mbox{\textsc{PyRoot}}\xspace}
\def\photos     {\mbox{\textsc{Photos}}\xspace}

%% Languages
\def\cpp        {\mbox{\textsc{C\raisebox{0.1em}{{\footnotesize{++}}}}}\xspace}
\def\python     {\mbox{\textsc{Python}}\xspace}
\def\ruby       {\mbox{\textsc{Ruby}}\xspace}
\def\fortran    {\mbox{\textsc{Fortran}}\xspace}
\def\svn        {\mbox{\textsc{SVN}}\xspace}

%% Data processing
\def\kbytes     {\ensuremath{{\rm \,kbytes}}\xspace}
\def\kbsps      {\ensuremath{{\rm \,kbytes/s}}\xspace}
\def\kbits      {\ensuremath{{\rm \,kbits}}\xspace}
\def\kbsps      {\ensuremath{{\rm \,kbits/s}}\xspace}
\def\mbsps      {\ensuremath{{\rm \,Mbits/s}}\xspace}
\def\mbytes     {\ensuremath{{\rm \,Mbytes}}\xspace}
\def\mbps       {\ensuremath{{\rm \,Mbyte/s}}\xspace}
\def\mbsps      {\ensuremath{{\rm \,Mbytes/s}}\xspace}
\def\gbsps      {\ensuremath{{\rm \,Gbits/s}}\xspace}
\def\gbytes     {\ensuremath{{\rm \,Gbytes}}\xspace}
\def\gbsps      {\ensuremath{{\rm \,Gbytes/s}}\xspace}
\def\tbytes     {\ensuremath{{\rm \,Tbytes}}\xspace}
\def\tbpy       {\ensuremath{{\rm \,Tbytes/yr}}\xspace}

\def\dst        {DST\xspace}

%%%%%%%%%%%%%%%%%%%%%%%%%%%
% Detector related
%%%%%%%%%%%%%%%%%%%%%%%%%%%

%% Detector technologies
\def\nonn {\ensuremath{\rm {\it{n^+}}\mbox{-}on\mbox{-}{\it{n}}}\xspace}
\def\ponn {\ensuremath{\rm {\it{p^+}}\mbox{-}on\mbox{-}{\it{n}}}\xspace}
\def\nonp {\ensuremath{\rm {\it{n^+}}\mbox{-}on\mbox{-}{\it{p}}}\xspace}
\def\cvd  {CVD\xspace}
\def\mwpc {MWPC\xspace}
\def\gem  {GEM\xspace}

%% Detector components, electronics
\def\tell1  {TELL1\xspace}
\def\ukl1   {UKL1\xspace}
\def\beetle {Beetle\xspace}
\def\otis   {OTIS\xspace}
\def\croc   {CROC\xspace}
\def\carioca {CARIOCA\xspace}
\def\dialog {DIALOG\xspace}
\def\sync   {SYNC\xspace}
\def\cardiac {CARDIAC\xspace}
\def\gol    {GOL\xspace}
\def\vcsel  {VCSEL\xspace}
\def\ttc    {TTC\xspace}
\def\ttcrx  {TTCrx\xspace}
\def\hpd    {HPD\xspace}
\def\pmt    {PMT\xspace}
\def\specs  {SPECS\xspace}
\def\elmb   {ELMB\xspace}
\def\fpga   {FPGA\xspace}
\def\plc    {PLC\xspace}
\def\rasnik {RASNIK\xspace}
\def\elmb   {ELMB\xspace}
\def\can    {CAN\xspace}
\def\lvds   {LVDS\xspace}
\def\ntc    {NTC\xspace}
\def\adc    {ADC\xspace}
\def\led    {LED\xspace}
\def\ccd    {CCD\xspace}
\def\hv     {HV\xspace}
\def\lv     {LV\xspace}
\def\pvss   {PVSS\xspace}
\def\cmos   {CMOS\xspace}
\def\fifo   {FIFO\xspace}
\def\ccpc   {CCPC\xspace}

%% Chemical symbols
\def\cfourften     {\ensuremath{\rm C_4 F_{10}}\xspace}
\def\cffour        {\ensuremath{\rm CF_4}\xspace}
\def\cotwo         {\ensuremath{\rm CO_2}\xspace} 
\def\csixffouteen  {\ensuremath{\rm C_6 F_{14}}\xspace} 
\def\mgftwo     {\ensuremath{\rm Mg F_2}\xspace} 
\def\siotwo     {\ensuremath{\rm SiO_2}\xspace} 

%%%%%%%%%%%%%%%
% Special Text 
%%%%%%%%%%%%%%%
\newcommand{\eg}{\mbox{\itshape e.g.}\xspace}
\newcommand{\ie}{\mbox{\itshape i.e.}}
\newcommand{\etal}{{\slshape et al.\/}\xspace}
\newcommand{\etc}{\mbox{\itshape etc.}\xspace}
\newcommand{\cf}{\mbox{\itshape cf.}\xspace}
\newcommand{\ffp}{\mbox{\itshape ff.}\xspace}
\newcommand{\wrt}{\mbox{\itshape w.r.t.}\xspace}

 % Add in the predefined LHCb symbols
%%%%%%%%%%%%%%%%%%%%%%%%%
% restore from template
\def\ellell     {\ensuremath{\ell^+ \ell^-}\xspace}
\def\hlt    {HLT\xspace}
\def\hltone {HLT1\xspace}
\def\hlttwo {HLT2\xspace}
%%%%%%%%%%%%%%%%%%%%%%%%%
%  Alex\rq{}s stuff
\def\Kstarzz  {\ensuremath{\kaon^{*0}_{0}}\xspace}
\def\Kstarzo  {\ensuremath{\kaon^{*0}_{1}}\xspace}
\def\Kstarzt  {\ensuremath{\kaon^{*0}_{2}}\xspace}
\def\Kstarzj  {\ensuremath{\kaon^{*0}_{J}}\xspace}
%% Angular observables
\def\FL    {\ensuremath{F_{\mathrm{L}}}\xspace}
\def\FT    {\ensuremath{F_{\mathrm{T}}}\xspace}
\def\AIm   {\ensuremath{A_{\mathrm{Im}}}\xspace}
\def\FSl   {\ensuremath{F_{\mathrm{S}}^{l}}\xspace}
\def\FTl   {\ensuremath{F_{\mathrm{T}}^{l}}\xspace}
\def\ASl   {\ensuremath{A_{\mathrm{S}}^{l}}\xspace}

\def\OS#1     {\ensuremath{S_{#1}}\xspace}           % 3
\def\OA#1     {\ensuremath{A_{#1}}\xspace}           % 9
\def\J#1     {\ensuremath{J_{#1}}\xspace}           % 9
\def\I#1     {\ensuremath{I_{#1}}\xspace}           % 9
\def\RJ#1     {\ensuremath{R_{#1}^J}\xspace}           % 9
\def\RI#1     {\ensuremath{R_{#1}^I}\xspace}           % 9

\def\Ap {\ensuremath{A_{||}}\xspace}
\def\Ab {\ensuremath{A_{\bot}}\xspace}
\def\Az {\ensuremath{A_{0}}\xspace}

\def\fs{\ensuremath{f_s}}
\def\fd{\ensuremath{f_d}}
%
\def\kpi                {\ensuremath{K\pi}\xspace}

\def\ktopi               {\ensuremath{K\leftrightarrow\pi}\xspace}

% COMBOS\def\kstmm    {\Kstarz\mumu\xspace}
\def\kpimm    {\ensuremath{K\pi\mup\mun}\xspace}
\def\kstll    {\ensuremath{\Kstarz\ellell}\xspace}
\def\kpill    {\ensuremath{\kpi\ellell}\xspace}
% MASSES
\def\mB    {\ensuremath{m_{\Bd}	}\xspace}
\def\mkpi              {\ensuremath{m_{K\pi}}\xspace}
\def\mmm              {\ensuremath{m_{\mumu}}\xspace}
\def\mkstmm    {\ensuremath{m_{\kstmm}}\xspace}
\def\mkpimm    {\ensuremath{m_{\kpimm}}\xspace}
\def\mkpill            {\ensuremath{m_{\kpill}}\xspace}

%
\def\BdToKstll    {\decay{\Bd}{\Kstarz\ellell}}
\def\BdbToKstbll   {\decay{\Bdb}{\Kstarzb\ellell}}
\def\BdToKpill    {\decay{\Bd}{\Kp\pim\ellell}}
\def\BdbToKpill    {\decay{\Bdb}{\Km\pip\ellell}}
\def\BdToKpimm    {\decay{\Bd}{\Kp\pim\mumu}}
\def\BdbToKpimm    {\decay{\Bdb}{\Km\pip\mumu}}
\def\BdToJpsiKpi    {\decay{\Bd}{\jpsi\Kp\pim}}
\def\BdbToJpsiKpi    {\decay{\Bdb}{\jpsi\Km\pip}}
\def\BdToJpsiPhi    {\decay{\Bd}{\jpsi\phi}}
\def\BdToKstzmm    {\decay{\Bd}{\Kstarzz\mumu}}
\def\BdToKstomm    {\decay{\Bd}{\Kstarzo\mumu}}
\def\BsToJpsiKpi    {\decay{\Bs}{\jpsi\Kp\pim}}
%


%
\def\phiacp {\ensuremath{\phi_{\mathrm{ACP}}}\xspace}
\def\phiprime {\ensuremath{\phi^{'}}\xspace}
\def\phiprimecp {\ensuremath{\phi^{'}_{\ACP}}\xspace}
% Observables for S+P+D
\def\FS    {\ensuremath{F_{\mathrm{S}}}\xspace}
\def\FSi    {\ensuremath{\mathcal{F}_{\mathrm{S}}}\xspace}
\def\FP    {\ensuremath{F_{\mathrm{P}}}\xspace}
\def\FPi   {\ensuremath{\mathcal{F}_{\mathrm{P}}}\xspace}
\def\FD    {\ensuremath{F_{\mathrm{D}}}\xspace}
\def\FDi   {\ensuremath{\mathcal{F}_{\mathrm{D}}}\xspace}


%normalisations
\def\NP   {\ensuremath{\mathcal{N}_{\mathrm{1}}}\xspace}
\def\NS   {\ensuremath{\mathcal{N}_{\mathrm{0}}}\xspace}
\def\ND   {\ensuremath{\mathcal{N}_{\mathrm{2}}}\xspace}

% combinations
\def\FSD    {\ensuremath{F_{\mathrm{SD}}}\xspace}
\def\FSP    {\ensuremath{F_{\mathrm{SP}}}\xspace}
\def\FPD    {\ensuremath{F_{\mathrm{PD}}}\xspace}
% longitudinal components
\def\FLD    {\ensuremath{F_{\mathrm{L}}^{\mathrm{D}}}\xspace}
\def\FLS    {\ensuremath{F_{\mathrm{L}}^{\mathrm{S}}}\xspace}
\def\FLP    {\ensuremath{F_{\mathrm{L}}^{\mathrm{P}}}\xspace}
% transverse components
\def\FTD    {\ensuremath{F_{\mathrm{T}}^{\mathrm{D}}}\xspace}
\def\FTP    {\ensuremath{F_{\mathrm{T}}^{\mathrm{P}}}\xspace}
% asymmetries
\def\AS    {\ensuremath{A_{\mathrm{S}}}\xspace}
\def\ASi    {\ensuremath{\mathcal{A}_{\mathrm{S}}}\xspace}
\def\ASP    {\ensuremath{A_{\mathrm{SP}}}\xspace}
\def\AD    {\ensuremath{A_{\mathrm{D}}}\xspace}
\def\ASD    {\ensuremath{A_{\mathrm{SD}}}\xspace}
\def\ASDi    {\ensuremath{\mathcal{A}_{\mathrm{SD}}}\xspace}
\def\APD    {\ensuremath{A_{\mathrm{PD}}}\xspace}
\def\APDi    {\ensuremath{\mathcal{A}_{\mathrm{PD}}}\xspace}
\def\APDPa    {\ensuremath{A_{\mathrm{PD}}^{\mathrm{T1}}}\xspace}
\def\APDPb    {\ensuremath{A_{\mathrm{PD}}^{\mathrm{T2}}}\xspace}
\def\APDL    {\ensuremath{A_{\mathrm{PD}}^{\mathrm{L}}}\xspace}
\def\APDT    {\ensuremath{A_{\mathrm{PD}}^{\mathrm{T}}}\xspace}
\def\ADP    {\ensuremath{A_{\mathrm{||}}^{\mathrm{D}}}\xspace}
\def\ADT    {\ensuremath{A_{\mathrm{\bot}}^{\mathrm{D}}}\xspace}
\def\AFBD    {\ensuremath{A_{\mathrm{FB}}^{\mathrm{D}}}\xspace}
\def\AFBO    {\ensuremath{A_{\mathrm{FB}}^{\mathrm{1}}}\xspace}
\def\AFBT    {\ensuremath{A_{\mathrm{FB}}^{\mathrm{2}}}\xspace}

\def\psq    {\ensuremath{p^2}\xspace}

\def\ctlsq    {\ensuremath{\cos^2{\theta_l}}\xspace}
\def\ctksq    {\ensuremath{\cos^2{\theta_K}}\xspace}
\def\stl      {\ensuremath{\sin{\theta_l}}\xspace}
\def\stk      {\ensuremath{\sin{\theta_K}}\xspace}
\def\stlsq    {\ensuremath{\sin^2{\theta_l}}\xspace}
\def\stksq    {\ensuremath{\sin^2{\theta_K}}\xspace}

\def\dctl       {\ensuremath{\mathrm{dcos}{\theta_l}}\xspace}
\def\dctk       {\ensuremath{\mathrm{dcos}{\theta_K}}\xspace}


%decays
\def\dqsq       {{\deriv q^2}\xspace}


\def\BdJpsiKstar  {\decay{\Bd}{\jpsi\Kstarz}}
\def\BdoJpsiKstar {\decay{\Bdb}{\jpsi\Kstarzb}}
\def\BdPsiTwoSKstar   {\decay{\Bd}{\Kstarz\psi(2S)}}
\def\BsPhimm        {\decay{\Bs}{\phi\mumu}}
\def\BsPhiee        {\decay{\Bs}{\phi\epem}}
\def\BdKstee          {\decay{\Bd}{\Kstarz\epem}}
\def\BuKstmm       {\decay{\Bu}{\Kstarp\mumu}}
\def\LbLmm        {\decay{\Lb}{\Lambdares^{*}(1520)\mumu}}
\def\BuJpsiKstar  {\decay{\Bu}{\jpsi\Kstarzp}}

\def\BsToJpsiKstar  {\decay{\Bs}{\jpsi\Kstarz}}
\def\BsKstmm  {\decay{\Bs}{\Kstarz\mumu}}

\def\bdll  {\decay{\bquark}{\dquark\ellell}}

\def\Bupimm        {\decay{\Bu}{\pip\mumu}}
\def\Bdrhomm        {\decay{\Bd}{\rho\mumu}}

\def\bstt  {\decay{\bquark}{\dquark\tau\tau}}
\def\BdKsttt  {\decay{\Bs}{\Kstarz\tau\tau}}
\def\BuKtt  {\decay{\Bu}{\Kp\tau\tau}}



\def\Kstee          {\Kstarz\epem}
\def\Kstmm       {\Kstarp\mumu}

\def\varphi    {\ensuremath{\phi}\xspace}


\def\gevgevcccc    {\ensuremath{\gev^{2} / c^{4} }\xspace}
\def\mevmevcccc    {\ensuremath{\mev^{2} / c^{4} }\xspace}



\def\DLL {\ensuremath{\Delta\log\mathcal{L}}\xspace}
\def\dll {\ensuremath{\Delta\log\mathcal{L}}\xspace}
\def\dllmu {\ensuremath{\Delta\log\mathcal{L}_\mu}\xspace}
\def\dlle {\ensuremath{\Delta\log\mathcal{L}_e}\xspace}
\def\ismuon {\ensuremath{\text{IsMuon}}\xspace}





%RK stuff


\def\RK         {\ensuremath{R_{\mathrm{K}}}\xspace}

\def\Kll  {\ensuremath{\Kp\ellell}\xspace}
\def\Kee  {\ensuremath{\Kp\epem}\xspace}
\def\Kmm  {\ensuremath{\Kp\mumu}\xspace}
\def\JpsiKee  {\ensuremath{\jpsi(\epem)\Kp}\xspace}
\def\JpsiKmm  {\ensuremath{\jpsi(\mumu)\Kp}\xspace}
\def\Kpipi  {\ensuremath{\Kp\pip\pim}\xspace}

\def\BuJpsiKll  {\decay{\Bu}{\jpsi(\ellell)\Kp}}
\def\BuKll  {\decay{\Bu}{\Kp\ellell}}
\def\BuJpsiKmm  {\decay{\Bu}{\jpsi(\mumu)\Kp}}
\def\BuKmm  {\decay{\Bu}{\Kp\mumu}}
\def\BuJpsiKee  {\decay{\Bu}{\jpsi(\epem)\Kp}}
\def\BuKee  {\decay{\Bu}{\Kp\epem}}

\def\BdKll  {\decay{\Bd}{K_1\ellell}}
\def\BdKmm  {\decay{\Bd}{K_1\mumu}}
\def\BdKee  {\decay{\Bd}{K_1\epem}}

\def\BuKpipi  {\decay{\Bu}{\Kp\pip\pim}}

\def\BuJpsiKstll  {\decay{\Bu}{\jpsi(\ellell)\Kstarp}}
\def\BuKstll  {\decay{\Bu}{\Kstarp\ellell}}
\def\BuJpsiKstmm  {\decay{\Bu}{\jpsi(\mumu)\Kstarp}}
\def\BuKstmm  {\decay{\Bu}{\Kstarp\mumu}}
\def\BuJpsiKstee  {\decay{\Bu}{\jpsi(\epem)\Kstarp}}
\def\BuKstee  {\decay{\Bu}{\Kstarp\epem}}

\def\BdJpsiKstll  {\decay{\Bd}{\jpsi(\ellell)\Kstarz}}
\def\BdKstll  {\decay{\Bd}{\Kstarz\ellell}}
\def\BdJpsiKstmm  {\decay{\Bd}{\jpsi(\mumu)\Kstarz}}
\def\BdKstmm  {\decay{\Bd}{\Kstarz\mumu}}
\def\BdJpsiKstee  {\decay{\Bd}{\jpsi(\epem)\Kstarz}}
\def\BdKstee  {\decay{\Bd}{\Kstarz\epem}}


\def\BuJpsiXee {\decay{\Bu}{(\decay{\jpsi}{\epem})\Kp X}\xspace}
\def\BdJpsiXee {\decay{\Bd}{(\decay{\jpsi}{\epem})\Kp X}\xspace}
\def\BsJpsiXee {\decay{\Bs}{(\decay{\jpsi}{\epem})\Kp X}\xspace}
\def\LbJpsiXee {\decay{\Lb}{(\decay{\jpsi}{\epem})\Kp X}\xspace}


\def\bToccXee {\decay{\bquark}{(\decay{\ccbar}{\epem})\Kp X}\xspace}
\def\bToJpsiXee {\decay{\bquark}{(\decay{\jpsi}{\epem})\Kp X}\xspace}
\def\bToXee {\decay{\bquark}{{\epem}\Kp X}\xspace}
\def\bToccXmm {\decay{\bquark}{(\decay{\ccbar}{\mumu})\Kp X}\xspace}
\def\bToJpsiXmm {\decay{\bquark}{(\decay{\jpsi}{\mumu})\Kp X}\xspace}
\def\bToXmm {\decay{\bquark}{{\mumu}\Kp X}\xspace}
\def\bToccXll {\decay{\bquark}{(\decay{\ccbar}{\ellell})\Kp X}\xspace}
\def\bToJpsiXll {\decay{\bquark}{(\decay{\jpsi}{\ellell})\Kp X}\xspace}
\def\bToXmll {\decay{\bquark}{{\ellell}\Kp X}\xspace}


\def\Bsemu  {\decay{\Bs}{\mmu\electron}}

\def\mkee {\ensuremath{m_{\Kee}}\xspace}
\def\mjpsikee {\ensuremath{m_{\JpsiKee}}\xspace}





% TRIGGER LINES

\def\lzmuon   {L0Muon\xspace}
\def\hltotrack    {HLT1Track\xspace}
\def\hltomuon {HLT1TrackMuon\xspace}
\def\hlttwotopo {HLT2Topological\xspace}
\def\mutopo {HLT2MuTopo\xspace}
\def\muntrack {HLT2MuNTrack\xspace}
\def\muonetrack {HLT2Mu1Track\xspace}
\def\mutwotrack {HLT2Mu2Track\xspace}
\def\muthreetrack {HLT2Mu3Track\xspace}


\def\ipchisq {IP \ensuremath{\chi^2}\xspace}
\def\vtxchisq {vertex \ensuremath{\chi^2}\xspace}
\def\trchisq {track \ensuremath{\chi^2}\xspace}
\def\fdchisq {\ensuremath{\mathrm{flight\ distance}\ \chi^2}\xspace}

\def\probnnk {\texttt{ProbNNk}\xspace}
\def\probnnpi {\texttt{ProbNNpi}\xspace}
\def\probnnp {\texttt{ProbNNp}\xspace}
\def\probnne {\texttt{ProbNNe}\xspace}

\def\BtoDmunu {\ensuremath{\decay{\B}{\D\mu\nu}}\xspace}
\def\BtoDsmunu {\ensuremath{\decay{\B}{\Ds\mu\nu}}\xspace}

\def\kdllkpi     {\ensuremath{\kaon\Delta\log\mathcal{L}_{\kaon\pion}}\xspace}
\def\pidllkpi     {\ensuremath{\pion\Delta\log\mathcal{L}_{\kaon\pion}}\xspace}


\def\minctk  {\ensuremath{\cos{\theta_K^{\mathrm{min}}}}\xspace}
\def\maxctk  {\ensuremath{\cos{\theta_K^{\mathrm{max}}}}\xspace}



\def\offsel {\ensuremath{\mathrm{Offline~selected~events}}\xspace}
\def\genlvl {\ensuremath{\mathrm{Generator~level~events}}\xspace}

\def\CB {\ensuremath{\mathrm{CB}}\xspace}
\def\Spdf {\ensuremath{\mathrm{S}}\xspace}
\def\Bpdf {\ensuremath{\mathrm{B}}\xspace}

\def\RK         {\ensuremath{R_{\kaon}}\xspace}
\def\RH         {\ensuremath{R_{H}}\xspace}
\def\RP         {\ensuremath{R_{P}}\xspace}
\def\RV         {\ensuremath{R_{V}}\xspace}
\def\RKst         {\ensuremath{R_{\Kstar}}\xspace}
\def\Rphi         {\ensuremath{R_{\phi}}\xspace}
\def\RXs         {\ensuremath{R_{X_s}}\xspace}

\def\XH         {\ensuremath{\RH/\RK}\xspace}
\def\XKst         {\ensuremath{R_{\Kstar}/\RK}\xspace}
\def\Xphi         {\ensuremath{R_{$\phi$}/\RK}\xspace}
\def\XXs         {\ensuremath{R_{$X_s$}\RK}\xspace}

\def\RPV         {\ensuremath{\RV/\RP}\xspace}

%muon specific
\def\Cm#1      {\ensuremath{\mathcal{C}_{#1}^{\mu}}\xspace}                       % 9
\def\Cmp#1     {\ensuremath{\mathcal{C}_{#1}^{' \mu}}\xspace}                    % 7
\def\Cmeff#1   {\ensuremath{\mathcal{C}_{#1}^{\mathrm{(eff)} \mu}}\xspace}        % 9  
\def\Cmpeff#1  {\ensuremath{\mathcal{C}_{#1}^{'\mathrm{(eff)} \mu}}\xspace}       % 7
\def\Ompe#1    {\ensuremath{\mathcal{O}_{#1}^{\mu}}\xspace}                       % 2
\def\Ompep#1   {\ensuremath{\mathcal{O}_{#1}^{' \mu}}\xspace}                    % 7

\def\Cl#1      {\ensuremath{\mathcal{C}_{#1} (\ell)}\xspace}                       % 9
\def\Clp#1     {\ensuremath{\mathcal{C}_{#1}^{'}(\ell)}\xspace}                    % 7
\def\Cleff#1   {\ensuremath{\mathcal{C}_{#1}^{\mathrm{(eff)}}(\ell)}\xspace}        % 9  
\def\Clpeff#1  {\ensuremath{\mathcal{C}_{#1}^{'\mathrm{(eff)}}(\ell)}\xspace}       % 7
\def\Olpe#1    {\ensuremath{\mathcal{O}_{#1} (\ell)}\xspace}                       % 2
\def\Olpep#1   {\ensuremath{\mathcal{O}_{#1}\,^{'}(\ell)}\xspace}                    % 7

%
%electron specific
\def\Ce#1      {\ensuremath{\mathcal{C}_{#1}^{e}}\xspace}                       % 9
\def\Cep#1     {\ensuremath{\mathcal{C}_{#1}^{' e}}\xspace}                    % 7
\def\Ceeff#1   {\ensuremath{\mathcal{C}_{#1}^{\mathrm{(eff)} e}}\xspace}        % 9  
\def\Cepeff#1  {\ensuremath{\mathcal{C}_{#1}^{'\mathrm{(eff)} e}}\xspace}       %
\def\Oepe#1    {\ensuremath{\mathcal{O}_{#1}^{e}}\xspace}                       % 2
\def\Oepep#1   {\ensuremath{\mathcal{O}_{#1}^{' e}}\xspace}                    % 7



% to show numerical labels in the bibliography (default is to show no labels); only
%useful if you make citations in your resume
%\makeatletter
%\renewcommand*{\bibliographyitemlabel}{\@biblabel{\arabic{enumiv}}}
%\makeatother
%\renewcommand*{\bibliographyitemlabel}{[\arabic{enumiv}]}% CONSIDER REPLACING THE
%ABOVE BY THIS
%\usepackage{hyperref} % Hyperlinks in referencesor: Environment figure undefined.
%\usepackage[all]{hypcap}
% Uses hyperref to link DOI
%\newcommand\doilink[1]{\href{http://dx.doi.org/#1}{#1}}
%\newcommand\doi[1]{doi:\doilink{#1}}
%\usepackage{cite}
\usepackage{natbib}
\usepackage{bibentry}
%\makeatletter\let\saved@bibitem\@bibitem\makeatother
\usepackage{cite}
%\usepackage{mciteplus}
\nobibliography*
% bibliography with mutiple entries
%\usepackage{multibib}
%\newcites{book,misc}{{Books},{Others}}
%----------------------------------------------------------------------------------
%            content
%----------------------------------------------------------------------------------
\begin{document}
\bibliographystyle{Alex}
%-----       resume       ---------------------------------------------------------
\makecvtitle
\vspace{-1cm}

I am a highly motivated physicist with five years’ experience of contributing to the
\lhcb collaboration at \cern.
My research is focussed on searching for physics beyond the standard model in \bsll
decays.

\section{Employment history}
\cventry{2013 to 2015}{Post doctoral researcher}{\tud}{Germany}{}{\normalsize 
Post-doctoral position in experimental particle physics as part of the \lhcb collaboration in the Emmy Noether group headed by Johannes Albrecht. 
}

\section{Education}
\cventry{Oct 2013}{PhD, High Energy Physics}{\icl}{UK}{}{\normalsize 
Thesis: \href{https://cds.cern.ch/record/1607078?ln=en}{\emph{Exploring $b\rightarrow s$
electroweak penguins at LHCb}},
Supervisor: \href{http://www3.imperial.ac.uk/people/u.egede}{Prof. Ulrik Egede} \\
Research PhD searching for physics beyond the Standard Model on the \lhcb experiment
at CERN.
}
\cventry{Jun 2009}{MSci (Hons), Physics With Theoretical Physics}{\icl}{UK}{}
{\normalsize First Class degree concentrating on the theoretical aspects of physics,
specifically to understand current research into particle physics and cosmology.
This four year course involved specific modules in applied mathematics, statistics
and computing dedicated to implementing algorithms for modelling and data analysis.}
\cventry{Aug 2005}{A-levels, GCSEs}{Hardenhuish School}{Wiltshire, UK}
{}{A-levels: Physics (A), Mathematics (A), Chemistry (A), Further Mathematics (A). \\
GCSEs: 3 A*, 3 A, 3 B.}

\section{Research Experience}
\cvitem{2015 to 2017}{\textbf{Physics analysis group convener} As a convener of \lhcb’s {\textit electroweak penguin} working group, I coordinate research projects internationally and am responsible for thirty researchers ranging from students to faculty members.
}
\cvitem{2013 to 2014}{\textbf{Lead analyst on testing lepton universality} %Most recently, I have tested lepton universality using \BuKll decays.
The measurement of \RK, the ratio of \BuKmm to \BuKee decays, was not expected to be possible at \lhcb.
In October 2013, I took over as the project lead and developed new models to describe the data, 
implemented the calculations in a coherent framework and brought the result to publication. 
This measurement was published in PRL and selected as a \textit{Viewpoint in Physics}, one of four from \lhcb.
}
\cvitem{2010 to 2012}{ \textbf{Experimental analyst}
The decay \BdKstmm is one to the priority measurements for \lhcb.
Worked on the first data from the \lhc for three analyses of \BdKstmm, making the world’s most precise measurements. 
These measurements are an area of significant discussion in both the experimental and theoretical community.
}
\cvitem{2011 \& 2014}{ \textbf{Theoretical analyses} 
The existence of a \kpi S-wave is a critical aspect to precision measurements of \BdKstmm and
I performed one of the calculations the effect of the S-wave on experimental measurements.
At Dortmund, I initiated a collaboration with theorists to predict the size of the \kpi S-wave.
}
\cvitem{2010 \& 2014}{\textbf {Trigger development}
The prospects of new physics in \RK and \BdKstmm led me to develop the electron triggers for Run II of the \lhc.
Based on my experience developing the muon triggers for Run I, 
I have ensured that there will be sufficient rate of electron decays for precision studies to test lepton non-universality.
%This required a good understand of the \lhcb software framework, both in C++ and Python, 
%and the ability to work in a rapidly changing environment.	
}

\cventry{Summer 2008}{Undergraduate research placement}{\icl}{UK}{}{\normalsize 
The Ganga project has developed front-end software that allows hundreds of researchers to use many distributed computing systems across the world in a coherent format and is the main user software for \lhcb and \atlas.
Developed and integrated autonomous remote testing for the Ganga project and added reporting options to show test failure differences between different versions. 
Worked with established Python framework as part of a small team of 10 developers to implement the remote testing.
}

\cventry{Summer 2006 \& 2007}{Junior engineer}{Westinghouse Rail
Systems}{Wiltshire, UK}{}{\normalsize 
As a scholarship given to the best 3 students from local schools, 
worked as the sole data analyst for the first live railway trial of a multi-million
pound project.
Invited back for a second year to develop software to test the integration of a 
new railway track-side communications protocol.}

\section*{Teaching Experience}
\cvitem{2014}{ \textbf{English speaking tutorial group}, BSc particle physics, \tud.}
\cvitem{Autumn 2014}{ \textbf{Lecture}, Particle Identification, MSc particle detectors joint lecture course, \tud. }
\cvitem{Summer 2014}{ \textbf{Project supervision}, BSc and MSc final year projects, \tud. }
\cvitem{Winter 2014}{ \textbf{Lecture},  Particle identification, MSc particle detectors joint lecture course, \tud. }
\cvitem{2012}{ \textbf{Lab demonstrator}, BSc computational physics, \icl. }
\cvitem{2011}{ \textbf{Lab demonstrator}, BSc experimental physics, \icl.}

\section{Skills}
\cvitem{Key skills}{Problem solving, Data Analysis, Programming}
\cvdoubleitem{Computing}{\cpp, \python, \fortran}{Frameworks}{ROOT, boost, gsl, numpy/scipy}
\cvdoubleitem{OS}{Linux, Windows}{Tools}{SVN, Git, MS Office, \LaTeX, Vim}
\cvdoubleitem{Languages}{English (native), German (working)}{Additional}{Full, clean UK driving licence}


%\section{Languages}
%\cventry{year--year}{Job title}{Employer}{City}{}{Description line
%1\newline{}Description line 2}
%\cvitemwithcomment{Language 1}{Skill level}{Comment}
%\cvitemwithcomment{Language 2}{Skill level}{Comment}
%\cvitemwithcomment{Language 3}{Skill level}{Comment}
\section{References}
\cvitem{}{Available on request}


\clearpage
%
%
\section*{Publications}
\subsection*{\lhcb peer-reviewed papers} 
\cvitem{}{Publications with a significant contribution}
\publication{\bibentry{pimumu} (in review).	} 
\publication{\bibentry{LHCb-PAPER-2014-024}}
\publication{\bibentry{LHCb-PAPER-2013-037}}
\publication{\bibentry{LHCb-PAPER-2013-019}}
\publication{\bibentry{LHCb-PAPER-2011-020}}
\cvitem{}{ Additional author on more than 200 papers as a member of the LHCb collaboration.}
\subsection*{Peer-reviewed papers outside \lhcb}
\publication{\bibentry{Das:2014sra}}
\publication{\bibentry{Blake:2012mb}}
\subsection*{\lhcb conference submissions}
\publication{\bibentry{LHCb-CONF-2012-008}}
\publication{\bibentry{LHCb-CONF-2011-038}}
\subsection*{\lhcb documents}
\publication{\bibentry{Shires:1951309}}


\section*{Invited Talks} 
\subsection{International conferences}
\publication{\bibentry{Shires:AspenTalk}}
\publication{\bibentry{Shires:1459485}}
\subsection{Seminars}
\publication{\bibentry{Shires:Heidelberg}}
\publication{\bibentry{Shires:DortmundTalk2}}
\publication{\bibentry{Shires:CERNTalk}}
\publication{\bibentry{Shires:BonnTalk}}
\publication{\bibentry{Shires:DortmundTalk}}
\publication{\bibentry{ShiresIOP}}


\clearpage

%\begin{cvcolumns}
% \cvcolumn{Category 1}{\begin{itemize}\item Person 1\item Person 2\item Person
% 3\end{itemize}}
% \cvcolumn{Category 2}{Amongst others:\begin{itemize}\item Person 1, and\item
%Person 2\end{itemize}(more upon request)}
% \cvcolumn[0.5]{All the rest \& some more}{\textit{That} person, and \textbf{those}
%also (all available upon request).}
%\end{cvcolumns}

% Publications from a BibTeX file without multibib
% for numerical labels:
\renewcommand{\bibliographyitemlabel}{\@biblabel{\arabic{enumiv}}}% CONSIDER MERGING
%WITH PREAMBLE PART
% to redefine the heading string ("Publications"): \renewcommand{\refname}{Articles}
%\nocite{*}

\nobibliography{cv,LHCb-PAPER,LHCb-CONF} % 'publications' is the name of a BibTeX file
%\clearpage
%%-----       letter       ---------------------------------------------------------%% recipient data
%\recipient{Company Recruitment team}{Company, Inc.\\123 somestreet\\some city}
%\date{January 01, 1984}
%\opening{Dear Sir or Madam,}
%\closing{Yours faithfully,}
%\enclosure[Attached]{curriculum vit\ae{}} % use an optional argument to use a
%string other than "Enclosure", or redefine \enclname
%\makelettertitle
%\dots
%\makeletterclosing
\end{document}