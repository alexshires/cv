\documentclass[11pt,a4paper,sans]{moderncv} % possible options include font size
%('10pt', '11pt' and '12pt'), paper size ('a4paper', 'letterpaper', 'a5paper',
%'legalpaper', 'executivepaper' and 'landscape') and font family ('sans' and 'roman')

% moderncv themes
\moderncvstyle{casual} % style options are 'casual' (default), 'classic', 'oldstyle'
%and 'banking'
\moderncvcolor{blue} % color options 'blue' (default), 'orange', 'green', 'red',
%'purple', 'grey' and 'black'
%\renewcommand{\familydefault}{\sfdefault} % to set the default font; use
%'\sfdefault' for the default sans serif font, '\rmdefault' for the default roman
%one, or any tex font name
\nopagenumbers{} % uncomment to suppress automatic page numbering for CVs longer
%than one page

% character encoding
\usepackage[utf8]{inputenc} % if you are not using xelatex ou lualatex, replace by
%the encoding you are using
%\usepackage{CJKutf8} % if you need to use CJK to typeset your resume in Chinese,
%Japanese or Korean

% adjust the page margins
%\usepackage[scale=0.75]{geometry}
\usepackage[scale=0.85]{geometry}
\setlength{\hintscolumnwidth}{2cm}
\AtBeginDocument{\recomputelengths}
%\setlength{\hintscolumnwidth}{3cm} % if you want to change the width of the column
%with the dates
%\setlength{\makecvtitlenamewidth}{10cm} % for the 'classic' style, if you want to
%force the width allocated to your name and avoid line breaks. be careful though, the
%length is normally calculated to avoid any overlap with your personal info; use this
%at your own typographical risks...

% personal data
\name{}{Dr Alexander Shires}
%\title{Alexander Shires' Resume} % optional, remove / comment the line if not
%wanted
\address{Ostenhellweg 56}{44135 Dortmund}{Germany}% optional, remove / comment the
%line if not wanted; the "postcode city" and "country" arguments can be omitted or
%provided empty
\phone[mobile]{+49~(173)~690~9175} % optional, remove / comment the line if not
%wanted; the optional "type" of the phone can be "mobile" (default), "fixed" or "fax"
\phone[fixed]{+44~(7799)~823~210}
%\phone[fax]{+3~(456)~789~012}
\email{a.shires@gmail.com} % optional, remove / comment the line if not wanted
%\homepage{www.johndoe.com} % optional, remove / comment the line if not wanted
%\social[linkedin]{alex-shires} % optional, remove / comment the line if not wanted
\social[twitter]{AlexanderShires} % optional, remove / comment the line if not
\social[github]{alexshires} % optional, remove / comment the line if not wanted
%\extrainfo{additional information} % optional, remove / comment the line if not
%\photo[64pt][0.4pt]{picture} % optional, remove / comment the line if not wanted;
%'64pt' is the height the picture must be resized to, 0.4pt is the thickness of the
%frame around it (put it to 0pt for no frame) and 'picture' is the name of the
%picture file
%\quote{Some quote} % optional, remove / comment the line if not wanted
%Publications
%\newcommand{\publication}[3]{\cventry{#1}{#2}{}{}{}{#3}}
\newcommand{\publication}[1]{\vspace{-0.3cm}\cvitem{}{#1}}

\def\tud{Technische Universit\"at Dortmund\xspace}
\def\icl{Imperial College London\xspace}
\def\name{Dr Alexander Shires}
% %%%%%%%%%%%%%%%%%%%%
%  for LHCb aliases
% %%%%%%%%%%%%%%%%%%%%
\usepackage{ifthen} % for conditional statements
\newboolean{pdflatex}
\setboolean{pdflatex}{true} % use this if using non-eps figures
\newboolean{articletitles}
\setboolean{articletitles}{true} % False removes titles in references
\newboolean{uprightparticles}
\setboolean{uprightparticles}{false} %Set to true to get roman particle symbols
\usepackage{amsmath} % Adds a large collection of math symbols
\usepackage{xspace} % To avoid problems with missing or double spaces after
% predefined symbold
\usepackage{bm}
\usepackage{amssymb}
\usepackage{amsfonts}
\usepackage{upgreek} % Adds in support for greek letters in roman typeset
\input{lhcb-symbols-def} % Add in the predefined LHCb symbols

% to show numerical labels in the bibliography (default is to show no labels); only
%useful if you make citations in your resume
%\makeatletter
%\renewcommand*{\bibliographyitemlabel}{\@biblabel{\arabic{enumiv}}}
%\makeatother
%\renewcommand*{\bibliographyitemlabel}{[\arabic{enumiv}]}% CONSIDER REPLACING THE
%ABOVE BY THIS
%\usepackage{hyperref} % Hyperlinks in referencesor: Environment figure undefined.
%\usepackage[all]{hypcap}
% Uses hyperref to link DOI
%\newcommand\doilink[1]{\href{http://dx.doi.org/#1}{#1}}
%\newcommand\doi[1]{doi:\doilink{#1}}
%\usepackage{cite}
\usepackage{natbib}
\usepackage{bibentry}
%\makeatletter\let\saved@bibitem\@bibitem\makeatother
\usepackage{cite}
%\usepackage{mciteplus}
\nobibliography*
% bibliography with mutiple entries
%\usepackage{multibib}
%\newcites{book,misc}{{Books},{Others}}
%----------------------------------------------------------------------------------
%            content
%----------------------------------------------------------------------------------
\begin{document}
\bibliographystyle{Alex}
%-----       resume       ---------------------------------------------------------
\makecvtitle
\vspace{-1cm}

A highly-motivated physicist, I have five years’ experience of contributing to the
\lhcb collaboration at \cern whilst working at \icl and \tud.
My research is focussed on searching for physics beyond the standard model in \bsll
decays.\\


\cvitem{Key Skills}{ Physics, Statistics, Data Analysis, Programming}

\section{Education}
\cventry{Oct 2013}{PhD, High Energy Physics}{\icl}{UK}{}{
Thesis: \href{https://cds.cern.ch/record/1607078?ln=en}{\emph{Exploring $b\rightarrow s$
electroweak penguins at LHCb}},
Supervisor: \href{http://www3.imperial.ac.uk/people/u.egede}{Prof. Ulrik Egede} \\
Research PhD searching for physics beyond the Standard Model on the \lhcb experiment
at CERN.
%Implemented data analysis code to calculate the acceptance correction, a critical
%requirement for the first two measurements of the \BdToKstmm decay at \lhcb.
%Developed trigger software algorithms to select .
%Whilst living in Geneva for 15 months, participated in the running of the \lhcb
%experiment in 2011.
}  % arguments 3 to 6 can be left empty
\cventry{Jun 2009}{MSci (Hons), Physics With Theoretical Physics}{\icl}{UK}{}
{First Class degree concentrating on the theoretical aspects of physics,
specifically to understand current research into particle physics and cosmology.
This four year course involved specific modules in applied mathematics, statistics
and computing dedicated to implementing algorithms for modelling and data analysis.}
\cventry{Aug 2005}{A-levels, GCSEs}{Hardenhuish School}{Wiltshire, UK}
{}{A-levels: Physics (A), Mathematics (A), Chemistry (A), Further Mathematics (A). \\
GCSEs: 3 A*, 3 A, 3 B.}

\section{Skills}
\cvitem{Professional}{
Delivered multiple high profile results during my PhD and post-doctoral position, working over timescales ranging from a few months to several years.  
Record of successful collaboration with multiple researchers across Europe in a highly matrix environment.
Requested by the lead physicist in \lhcb to independantly review a critical project and delivered a thorough review ahead of schedule.
As the sole post--doctoral researcher at my current position, I set a professional example to the post-graduate students. 
Communication of my work is vital to it's success and I have strong public presentation skills, developed while leading discussions at a number of top academic institutions across Europe. 
}
\cvitem{Software}{Designed and implemented software across the \lhcb collaboration, 
from local scripts to production code in the \lhcb software trigger, 
used by multiple research groups and required by the entire collaboration. 
Regularly implement effective code to deliver results with a scalable and maintainable ethos.}
\cvdoubleitem{Computing}{C++, Python, Fortran}{Frameworks}{ROOT, boost, gsl,
numpy/scipy}
\cvdoubleitem{OS}{Linux, Windows}{Tools}{SVN, Git, MS Office, \LaTeX, Vim}
\cvdoubleitem{Languages}{English, German}{Other}{Full, clean UK driving
licence}
\section{Professional Experience}
\cventry{June 2013 to present}{Post doctoral researcher}{\tud}{Germany}{}{
Post-doctoral position as an experimental researcher working on data from the \lhcb experiment.
In my first sole project I developed new models from scratch to describe the data, implemented the calculations in a coherent framework and brought the result to publication.
Alongside this, I initiated a collaboration with the theory department to produce a prediction of a previously unknown quantity, vital for future research in my area.
Subsequent placement at \cern for three months for on-site software development to prepare to the reactivation of the \lhc.
On my return to Dortmund, I brought two more long term projects to fruition.
}
\cventry{Jan 2012 -- Apr 2013}{PhD student}{\icl}{UK}{}{
Delivered one result in a small team and started a second as a sole project whilst writing my thesis.
%Published the second measurement of the angular distribution of \BdKstmm at \lhcb and set up the first measurement of the \kpi S-wave contribution to the \BdKstmm decay.
}
\cventry{Aug 2010 -- Dec 2011}{PhD student}{\cern}{Switzerland}{}{
Placement as part of my PhD studentship, lived in Geneva and worked at CERN.
Produced the first measurement of my PhD on the first data from the \lhc, 
developed software vital to the running of the experiment and participated in the running of the \lhcb experiment during data-taking in 2011.
%Produced on the first measurement of the angular distribution of \BdKstmm at \lhcb, developed the trigger software for \lhcb and 
}

\cventry{Summer\\ 2008}{Undergraduate research placement}{\icl}{UK}{}{Developed and
integrated autonomous remote testing for the Ganga project.
Developed reporting options to show test failure differences between different versions of the software. 
Worked with an established software framework as part of a small team to implement the developments.
%\end{itemize}
}

\cventry{Summer \\2006 \& 2007}{Junior engineer}{Westinghouse Rail
Systems}{Wiltshire, UK}{}{
As a scholarship given to the best 3 students from local schools, worked as the sole data analyst for the first live railway trial of a multi-million pound project.
Invited back for a second year to develop software to test the integration of a new railway track-side communications protocol.}

%\section{Languages}
%\cvitemwithcomment{Language 1}{Skill level}{Comment}
%\cvitemwithcomment{Language 2}{Skill level}{Comment}
%\cvitemwithcomment{Language 3}{Skill level}{Comment}

%\section{Additional Skills}
%
%
\section*{Publications}
\publication{\bibentry{Das:2014sra}}
\publication{\bibentry{LHCb-PAPER-2014-024}}
\publication{\bibentry{LHCb-PAPER-2013-019}}
\publication{\bibentry{Blake:2012mb}}
\publication{\bibentry{LHCb-PAPER-2011-020}}
\cvitem{}{ Additional author on more than 200 papers as a member of the LHCb collaboration.}\vspace{-0.3cm}

\section*{Invited Talks} 
%\publication{\bibentry{Shires:Heidelberg}}
%\publication{\bibentry{Shires:DortmundTalk2}}
\publication{\bibentry{Shires:CERNTalk}}
\publication{\bibentry{Shires:AspenTalk}}
%\publication{\bibentry{Shires:BonnTalk}}
%\publication{\bibentry{Shires:DortmundTalk}}
%\publication{\bibentry{Shires:1459485}}
%\publication{\bibentry{ShiresIOP}}
\cvitem{}{Additional regular seminars at UK and German institutions}.

%\section*{Teaching Experience}
%\cvitem{ 2014 }{ English speaking tutorial group, $4^{\text{th}}$ particle physics, \tud.}
%\cvitem{ 2014 }{ Project supervision, $3^{\text{rd}}$ and $4^{\text{th}}$ year undergraduate course, \tud. }
%\cvitem{ 2014 }{ Particle identification seminar, part of the $4^{\text{th}}$ year
%particle detectors lecture course, \tud. }
%\cvitem{ 2012 }{ Computational lab demonstrator, $3^{\text{rd}}$ year undergradudate
%course, \icl. }
%\cvitem{ 2011 }{ Experimental lab demonstrator, $3^{\text{rd}}$ year undergraduate
%couse, \icl.}


%\cvitem{hobby 3}{Description}
%
%\section{Extra 1}
%\cvlistitem{Item 1}
%\cvlistitem{Item 2}
%\cvlistitem{Item 3. This item is particularly long and therefore normally spans
%over several lines. Did you notice the indentation when the line wraps?}
%
%\section{Extra 2}
%\cvlistdoubleitem{Item 1}{Item 4}
%\cvlistdoubleitem{Item 2}{Item 5\cite{book1}}
%\cvlistdoubleitem{Item 3}{Item 6. Like item 3 in the single column list before,
%this item is particularly long to wrap over several lines.}

\section{References}
\cvitem{}{Available on request}
%\begin{cvcolumns}
% \cvcolumn{Category 1}{\begin{itemize}\item Person 1\item Person 2\item Person
% 3\end{itemize}}
% \cvcolumn{Category 2}{Amongst others:\begin{itemize}\item Person 1, and\item
%Person 2\end{itemize}(more upon request)}
% \cvcolumn[0.5]{All the rest \& some more}{\textit{That} person, and \textbf{those}
%also (all available upon request).}
%\end{cvcolumns}

% Publications from a BibTeX file without multibib
% for numerical labels:
\renewcommand{\bibliographyitemlabel}{\@biblabel{\arabic{enumiv}}}% CONSIDER MERGING
%WITH PREAMBLE PART
% to redefine the heading string ("Publications"): \renewcommand{\refname}{Articles}
%\nocite{*}

\nobibliography{cv,LHCb-PAPER} % 'publications' is the name of a BibTeX file

% Publications from a BibTeX file using the multibib package
%\section{Publications}
%\nocitebook{book1,book2}
%\bibliographystylebook{plain}
%\bibliographybook{publications} % 'publications' is the name of a BibTeX file
%\nocitemisc{misc1,misc2,misc3}
%\bibliographystylemisc{plain}
%\bibliographymisc{publications} % 'publications' is the name of a BibTeX file

%\clearpage
%%-----       letter       ---------------------------------------------------------%% recipient data
%\recipient{Company Recruitment team}{Company, Inc.\\123 somestreet\\some city}
%\date{January 01, 1984}
%\opening{Dear Sir or Madam,}
%\closing{Yours faithfully,}
%\enclosure[Attached]{curriculum vit\ae{}} % use an optional argument to use a
%string other than "Enclosure", or redefine \enclname
%\makelettertitle
%\dots
%\makeletterclosing

%\clearpage\end{CJK*} % if you are typesetting your resume in Chinese using CJK; the
%\clearpage is required for fancyhdr to work correctly with CJK, though it kills the
%page numbering by making \lastpage undefined
\end{document}


%% end of file `template.tex'.